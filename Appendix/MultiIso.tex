\chapter{Additional Information on Multi-Isolation}
\label{app:MultiIso}

\textbf{Note:} This section is taken verbatim from SUS-15-008 PAS \cite{CMS-PAS-SUS-15-008} (thanks to our colleagues!).

The charged leptons produced in decays of heavy particles, such as $\PW$ and $\cPZ$ bosons 
or SUSY particles, are typically spatially isolated from the hadronic activity in the event,
while the leptons produced in the decays of hadrons or misidentified leptons are usually
embedded in jets. This distinction becomes less evident moving to
highly boosted systems where decay products tend to overlap.
%Therefore, in order to discriminate between the signal and background leptons, 
Therefore, given the higher collision energy explored in this analysis, a new and improved isolation definition is constructed using three variables as input:

\begin{itemize}
\item A mini-isolation (\miniIso)~\cite{Rehermann:2010vq} which is computed as a ratio of the scalar sum of transverse momenta 
of the charged hadrons, neutral hadrons, and photons within a cone of radius $R (\pt^\ell)$ around the lepton 
candidate direction at the origin, to the transverse momentum of the candidate. The cone radius $R$
depends on lepton $\pt$ as 
 \begin{equation}
R(\pt^\ell) = \dfrac{10\GeV}{\min\left[\max\left(\pt^\ell, 50\GeV\right), 200\GeV\right]}.
  \end{equation}
%The impact of the particles from other collisions in the event (pile up) is mitigated 
The varying isolation cone definition takes into account the aperture of b hadron decays as a function of
their $\pt$, and reduces the inefficiency from accidental overlap between the lepton and jets 
in a busy event environment.
\item A ratio between the lepton $\pt^\ell$ and $\pt^\text{jet}$ of a jet containing the lepton: 
 \begin{equation}
\ptRatio = \dfrac{\pt^\ell}{\pt^\text{jet}}.
  \end{equation}
The \ptRatio variable acts as a relative isolation in 
a larger cone. It improves mini-isolation performance when there
are no nearby jets, expecially for low-$\pt$ leptons. %It helps to estimate level of lepton isolation for low-$\pt$ leptons.
\item Transverse momentum of the lepton relative to the residual momentum of the closest jet after lepton momentum subtraction:
  \begin{equation}
    \ptRel=\frac{(\vec{p}(\text{jet})-\vec{p}(\ell))\cdot \vec{p}(\ell) }{|\vec{p}(\text{jet})-\vec{p}(\ell)|}.
  \end{equation}
The \ptRel variable allows to identify leptons that accidentally overlap with other jets in the event.
\end{itemize}

A lepton is considered to be isolated if the following condition is respected:
\begin{equation}
  \miniIso < I_1 \wedge ( \ptRatio > I_2 \vee \ptRel > I_3 )
\end{equation}

The values of $I_\text{i}, i = 1,2,3,$ depend on the lepton flavor; as the probability to misidentify a lepton is higher for electrons, 
tighter isolation values are used in this case (see Table~\ref{tab:isoWPs}). 
%The \multiIso working point for loose leptons is 
%used for the selection of the control regions and leptons used to form a veto.
%----------------------------------------------------------------------------------------------------------------------------
\begin{table}[h]
  \begin{center}
    %\small
    \caption{\label{tab:isoWPs}Multi-isolation working points used in the analysis.}
    \begin{tabular}{l|c|c|c}
      \hline
      Isolation value & Loose leptons  & Tight muons & Tight electrons  \\
      \hline\hline
      $I_1$ & 0.4 & 0.16 & 0.12 \\
      $I_2$ & 0 & 0.76 & 0.80 \\
      $I_3 (\GeV)$ & 0 & 7.2 & 7.2 \\
      \hline
    \end{tabular}
  \end{center}
\end{table}
%----------------------------------------------------------------------------------------------------------------------------
