\chapter{Relative Sensitivity of Signal Regions}
\label{app:RelativeSensitivity}

We define $r$ to be the ratio of the cross section limit and the signal production cross section. If $r < 1$, the signal hypothesis is excluded.

To get a feeling on how different signal regions contribute to the limits presented in Sec.~\ref{sec:Interpretation}, we present the expected single-channel $r$-values for the top 10 channels in Table~\ref{tab:topSensitivity}, using the signal with $m_\Sigma = 420\,\GeV$.

\begin{table}[h]
\centering
\caption{Relative Sensitivity of Signal Regions} \label{tab:topSensitivity}
\caption*{DY0: no OSSF pair, DYz1: 1 pair on \Z,\\ DYh1: 1 pair above \Z, DYgt0: $\geq 1$ pair}
\begin{tabular}{l c}
\hline\hline
Signal region & $r_\textrm{exp}$\\
\hline
L3 DYh1 LTMET550to750 & 2.6953\\
L3 DY0 LTMET550to750 & 3.6094\\
L3 DYh1 LTMET750to950 & 4.0781\\
L4 DYgt0 LTMET550to750 & 4.1094\\
L4 DYgt0 LTMET750to950 & 5.5938\\
\hline
L3 DY0 LTMET750to950 & 5.8750\\
L3 DYh1 LTMET350to550 & 6.2500\\
L3 DYz1 LTMET550to750 & 6.7188\\
L3 DY0 LTMET350to550 & 7.5938\\
L3 DYh1 LTMET950to1150 & 8.7812\\
\end{tabular}
\end{table}

As can be seen from the table, no single signal bin is sensitive enough to exclude the signal hypothesis; still, the table presents the relative sensitivities. The combination of several of these channels is sensitive enough to exclude a considerable range of the signal mass paramter (see Sec.~\ref{sec:Interpretation}).
