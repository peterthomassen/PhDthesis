\section{\texorpdfstring{\Z}{Z} + jets Background}
\label{sec:bkg_fakeLight}

The \Z + jets process contributes about 17\,\% of the total background. Since the misidentified leptons are not modeled with sufficient precision by simulation, we employ a data-driven method that uses correlated objects in order to predict the \Z + jets background.

In order to determine the background with misidentified electrons and muons from jets, we select events with the same selection as the signal except requiring one less lepton and requiring an isolated track instead. The isolation criteria that we require these tracks to satisfy are identical to our muon isolation criteria. We verify that the kinematic properties of the isolated tracks resemble those of the misidentified leptons.

The number of 3$\ell$ events in data per 2$\ell$ + track event in this sample then gives the lepton misidentification rate, $\frac{N(3\ell)}{N(2\ell + \textrm{track)}}$. Since we subtract contributions from other backgrounds in the numerator and the denominator, it describes the number of misidentified leptons as a fraction of the number of 2$\ell$ + track events from all processes that are not modeled otherwise.

\label{sec:bkg_fakeLight/jets}
The misidentification rate is measured using events with 3 electrons or muons including an OSSF pair on-\Z and $\MET < 50\,\GeV$. This is the prominent \Z peak region with an additional lepton. In this region, we measure the ratio of the number of 3$\ell$ and 2$\ell$+track events (misidentification rate).

We find the electron and muon misidentification rates to be $(1.59 \pm 0.15\stat)\,\%$ and $(1.49 \pm 0.13\stat)\,\%$, respectively. We apply a systematic uncertainty of 14\,\% to cover the variation of observed misidentification rates as a function of the flavor of the remaining prompt lepton pair in the event.

To apply the misidentification rate in a signal region, we multiply it by the total number of events found in the corresponding 2$\ell$ + track region in data. However, since we use MC to obtain the misidentified contribution for the \ttbar background, we need to correct for double-counting. We thus subtract the contribution from 2$\ell$ + track events as predicted by the \ttbar MC from the data.
%In rare cases due to statistical fluctuations, the subtraction might yield a (small) negative number. If that happens, we replace it by zero, to make sure that the background prediction behaves physically reasonably.


\subsection{Method}
\label{sec:bkg_fakeLight}

To determine the background with fake electrons and muons, we rely on looser objects measured in data that are emitted in a similar way in the decay chain and are therefore expected to be correlated with the fake leptons, and use them as lepton proxies.\footnote{These looser objects are not necessarily leptons as well. For example, a photon that converts into two leptons, one of which has very low \pt, may have kinematics which are very similar to the ones of the other conversion lepton that carries most of the \pt. (Of course, the selection of such objects may be tricky.)} We verify that the kinematic properties of these proxies resemble those of the fake leptons. We then generate a fake sample based on the 2$\ell$+[proxy object] data, treating the proxy objects as leptons (``seed sample''). Further down in the analysis chain, these fake leptons appear just as regular leptons (\eg when computing invariant masses). Proxy objects that can take multiple roles are considered the appropriate number of times (see Sec.~\ref{sec:bkg_fakeLight/jets}).

The number of 3$\ell$ events in data per 2$\ell$+[proxy object] event in this fake sample is then evaluated (``fake rate''). With the help of the fake rate, we predict the background in our signal regions, by applying it to the corresponding seed sample which requires one less lepton and a proxy object instead. Because the proxy objects appear as leptons, this is simply done by selecting the signal region from the fake sample.

To compute the fake rate $\frac{N(3\ell)}{N(2\ell + \textrm{[proxy object])}}$, we subtract contributions from other backgrounds in the numerator and the denominator. This step interacts with the MC background normalizations and thus requires an iterative process to converge. The fake rate then describes the number of fake leptons as a fraction of the number of 2$\ell$+[proxy object] events from all processes that have not been modeled otherwise.

When we apply the fake rate in a signal region, we multiply it by the total number of 2$\ell$+[proxy object] events found in the corresponding seed region in data, without any subtractions from the data sample. However, we use MC to obtain the fake contribution for certain backgrounds.\footnote{This is especially important for \ttbar when a b-tag is not present, since the fake rate is higher in \ttbar events, but there is no obvious way to discern these events from non-\ttbar events in the seed sample.} In these cases, double-counting needs to be mitigated. Therefore, we take the 2$\ell$+[proxy object] component of the background MC sample, apply the same fake rate as for data, and subtract the resulting prediction from the regular data-driven prediction (see \eg Sec.~\ref{sec:bkg_tt} for \ttbar). This is equivalent to keeping the seed sample clean of proxies originating from processes that are modeled otherwise. In rare cases due to statistical fluctuations, the subtraction might yield a (small) negative number. If that happens, we replace it by zero, to make sure that the background prediction behaves physically reasonably.
%\footnote{Another option would be to subtract the MC-fake-seed-driven background from the regular \ttbar MC prediction (again with a lower bound at 0). However, 8\,\TeV cross-checks have shown that this leads to less accurate results. Subtracting from the data-based prediction instead also seems more natural, as this amounts to a pruning of \ttbar-type events from the seed sample that we don't want to predict using the data-driven method.}

%We also study to what extent the fake rate depends on other properties of the event (for example the jet composition and spectra), and parameterize the fake rate as necessary. The freedom that we find in determining these parameterizations and kinematic weights is used to assess the systematic uncertainty of the background estimate.

\subsection{Fake leptons from asymmetric internal photon conversions (AIC)}
\label{sec:bkg_fakeLight/photons}
We look at the number of events that have 3 light leptons (no $\tau_\textrm{had}$) including an OSSF pair below \Z (\ie $m_{\ell\ell} < 81\,\GeV$), no b-tags, $\HT < 200\,\GeV$, and $\MET < 50\,\GeV$. This is essentially the \Z peak region, except that the dilepton invariant mass is not large enough to fall on the \Z peak, and a third lepton is present. This region primarily contains events from $\Z \to \ell\ell$ where one of the final state leptons radiates an off-shell photon which decays, or equivalently internally converts, asymmetrically to two additional leptons, one of which carries very low \pt and is not reconstructed as an independent object in the detector. The process of emission of an off-shell photon through asymmetric internal conversion then yields a single reconstructed lepton in the detector. Since the \pt of the lost lepton is low, the leading three leptons nearly reconstruct the invariant mass of the Z peak. The internal conversion process has an infrared singularity, so the distribution of off-shell photon masses is peaked at very low values. The resulting kinematic distribution in this region of phase space is then very similar to the emission of a real on-shell photon. 

We may therefore form a seed sample with photons as proxies for fake leptons coming from asymmetric internal conversion. All combinations are taken into account, \ie dilepton events with a photon enter the fake sample as four event types (two possible flavors, two possible charges). The photons are required to be within $dR = 0.30..0.60$ from another light lepton, as this is the characteristic distance for radiated photons of the type considered. \fixme{Show plot}

Looking in the seed sample, we find that the $2\ell+\gamma$ mass indeed reproduces the \Z peak, as shown in Fig.~\ref{fig:fakeLight_AIC_MLEPTONS}. %Note: Here, we use the \Z window range 75..100\,\GeV as the method does not model the $m_{\ell\ell\ell}$ shape correctly around 80\,\GeV (see Fig.~\ref{fig:fakeLight_AIC_MLEPTONS_fine}). \fixme{Study how that can be improved}

After applying additional corrections as described in the following paragraphs, we find that the photon fake rates are
\begin{itemize}
	\item muons: 1.60\,\% ($ee$ environment), 1.05\,\% ($\mu\mu$ environment),
	\item electrons: 3.5\,\% ($ee$ environment), 4.5\,\% ($\mu\mu$ environment).
\end{itemize}

\begin{figure}
\begin{center}
	\includegraphics[width=.5\textwidth]{Background/bkg_fakeLight/AIC_MLIGHTLEPTONS_muFake}%
	\includegraphics[width=.5\textwidth]{Background/bkg_fakeLight/AIC_MLIGHTLEPTONS_elFake}
	\caption{$m_{3\ell}$ distribution in AIC-dominated control region. \enskip left)~fake muon \enskip right)~fake electron
	\label{fig:fakeLight_AIC_MLEPTONS}}
\end{center}
\end{figure}

For photons faking muons, we find better agreement if we apply a loss factor of 0.8 to the photon \pt when creating the fake trilepton sample, attributing an average of 20\,\% of the \pt to the lost lepton. If this factor is not applied, the peak location is not modeled accurately.
%Note, however, that the width of the background peak is not the same as in data (see finely binned $m_{3\ell}$ distribution in Fig.~\ref{fig:fakeLight_AIC_MLEPTONS_fine}).

%\begin{figure}
%\begin{center}
%	\includegraphics[width=.5\textwidth]{Background/bkg_fakeLight/AIC_MLIGHTLEPTONS_muFake_fine}%
%	\includegraphics[width=.5\textwidth]{Background/bkg_fakeLight/AIC_MLIGHTLEPTONS_elFake_fine}
%	\caption{$m_{3\ell}$ distribution in AIC-dominated control region, fine binning. \enskip left)~fake muon \enskip right)~fake electron
%	\label{fig:fakeLight_AIC_MLEPTONS_fine}}
%\end{center}
%\end{figure}

Outside the trilepton \Z window, it is necessary to increase the fake rate by 1.8 to achieve agreement. The plot shown in this Section have this factor applied.

We apply a 52\,\% systematic uncertainty on the total photon-based background estimate. This is because the photon fake rate depends on the environment flavor within this range.


\subsection{Fake leptons from jets}
\label{sec:bkg_fakeLight/jets}
For fake electrons and muons from jets, our proxies are isolated tracks in the 2$\ell$ data sample. We produce a track-based fake 3$\ell$ background seed sample by re-assigning isolated tracks to the lepton collections. All combinations are taken into account, \ie tracks are used to create both a fake-$e$ and a fake-$\mu$ event.\footnote{Multiple fakes in an event are not considered (neither of same proxy type (\eg two tracks) nor of different type). Hybrid fakes (one from a track, one from a photon) are currently not supported for technical reasons; same-type fakes however turned out to cause problems with the \ttbar MC subtraction. Given the smallness of the fake rates ($O(10^{-2})$), the contribution from multiple fakes is negligible anyways.}

We then look at events with 3 light leptons (no $\tau_\textrm{had}$) including an OSSF pair on \Z, no b-tags, and $\MET < 50\,\GeV$. This is the prominent \Z peak region with an additional lepton.

To make the \pt distributions match and thus achieve more accurate background modeling, we apply weights to the track-based background in bins of the the lowest \pt lepton (proxy) which is generally the fake.
%Numbers from 8 TeV: For the electron part, the weights are between 0.4\fixme{Explain that this is so high because of light jets, but leptons come via semileptonic decays from c/s etc.} and 2.3 (10..25\,\GeV), and 4.0 above 25\,\GeV; for the muons, we only need to scale the first bin (10..15\,\GeV) by 1.13. [Fig.~\ref{fig:fakeLight_Z_MINleptonPT}]

\begin{figure}
\begin{center}
	\includegraphics[width=.5\textwidth]{Background/bkg_fakeLight/Z_muFake_MINMUONPT}%
	\includegraphics[width=.5\textwidth]{Background/bkg_fakeLight/Z_elFake_MINELECTRONPT}
	\caption{\pt distributions of the lowest \pt lepton. \enskip left)~fake muon \enskip right)~fake electron
	\label{fig:fakeLight_Z_MINleptonPT}}
\end{center}
\end{figure}

We then measure the ratio of the number of 3$\ell$ and 2$\ell$+track events (fake rate), and investigate the dependence of this fake rate on the flavor of both the fake lepton and of the \Z decay products.
\begin{itemize}
	\item muons: $(1.53 \pm 0.22)\,\%$ ($ee$ environment), $(1.49 \pm 0.17)\,\%$ ($\mu\mu$ environment),
	\item electrons: $(1.38 \pm 0.22)\,\%$ ($ee$ environment), $(1.77 \pm 0.19)\,\%$ ($\mu\mu$ environment).
\end{itemize}
We find that the rates are consistent within statistical uncertainties, and therefore use one combined number only: $(1.56 \pm 0.10)\,\%$

The statistical uncertainty of the fake rate is taken as a systematic uncertainty. An additional systematic uncertainty of 10\,\% is assigned to account for inaccuracies from the \pt weighting.

To show that the MC subtraction described in the introduction to this chapter is valid, we need to verify that the $n_\textrm{tracks}$ distribution in the \ttbar MC sample matches the one in data, so that we can trust the fake rate method used in data is applicable for the \ttbar MC subtraction; see Section \ref{sec:bkg_tt}.

Fig.~\ref{fig:fakeLight_Z_MOSSF} shows the mass distribution of the ``best'' OS dilepton pair across its full range in the trilepton control region, both with and without an AIC veto (for off-\Z events whose 3$\ell$ invariant mass is on \Z). Fig.~\ref{fig:fakeLight_Z_MOSSF_byFlavor} distinguishes by flavors. In case of ambiguity, the ``best'' OS dilepton pair is the one whose invariant mass is closest to the \Z mass, with the additional condition that pairs above the \Z window are not considered if there is a pair below the \Z window (thus shifting events from high-\Z to low-\Z, for a more separative background categorization). For a comparison with another \Z-ness binning scheme, see Appendix~\ref{app:Zbinning}.

\begin{figure}
\begin{center}
	\includegraphics[width=.5\textwidth]{Background/bkg_fakeLight/Z_MOSSF}%
	\includegraphics[width=.5\textwidth]{Background/bkg_fakeLight/Z_noAIC_MOSSF}
	\caption{$m_{\ell\ell}$ distribution in the dilepton + fake region. \enskip left)~including events with $m_{\ell\ell\ell}$ on-\Z \enskip right)~events with $m_{\ell\ell\ell}$ on-\Z removed from the dilepton-off-\Z regions
	\label{fig:fakeLight_Z_MOSSF}}
\end{center}
\end{figure}

\begin{figure}
\begin{center}
	\includegraphics[width=.5\textwidth]{Background/bkg_fakeLight/Z_1el2mu_MOSSF}%
	\includegraphics[width=.5\textwidth]{Background/bkg_fakeLight/Z_3el_MOSSF}\\
	\includegraphics[width=.5\textwidth]{Background/bkg_fakeLight/Z_3mu_MOSSF}%
	\includegraphics[width=.5\textwidth]{Background/bkg_fakeLight/Z_2el1mu_MOSSF}\\
	\caption{$m_{\ell\ell}$ distribution in the dilepton + fake region. \enskip top)~$e$ fake \enskip bottom)~$\mu$ fake; \enskip left)~$\Z \to \mu\mu$ \enskip right)~$\Z \to e e$
	\label{fig:fakeLight_Z_MOSSF_byFlavor}}
\end{center}
\end{figure}

As an additional cross-check, we show the \HT distribution in the \Z region (Fig.~\ref{fig:fakeLight_Z_HT}).

\begin{figure}
\begin{center}
	\includegraphics[width=.7\textwidth]{Background/bkg_fakeLight/Z_HT}
	\caption{$\HT$ distribution in the dilepton fake region (no OSSF pair mass cut)
	\label{fig:fakeLight_Z_HT}}
\end{center}
\end{figure}
