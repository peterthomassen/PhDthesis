\chapter{Introduction}
\label{sec:Introduction}

The Standard Model of particle physics (SM) \cite{i2003gauge,aitchison2003gauge,Peskin:257493} has been a grand success in describing the fundamental properties and interactions of elementary particles. Since its development in the second half of the 20th century, it has helped understand the basic principles of both how matter is formed on the microscopic level, and of how the universe\hairspace---\hairspace{}eventually a big agglomeration of interacting particles\hairspace---\hairspace{}behaves on the macroscopic scale as a consequence of the microscopic interactions.

While designed to describe experimental observations, the SM reaches beyond the mere reproduction experimental results. Its mathematical structure leads to predictions of what should be observed in experiments that have never been done before. To verify the validity of such predictions and thus of the concepts of the SM itself, the scientific community has built large accelerator machines such as the Large Hadron Collider (LHC) \cite{Evans:2008zzb} which pushes charged particles to very high velocities in order to have them collide at a defined point of interaction, around which detectors record the scatter particles with very high precision. This way, almost all SM predictions that have been put to an experimental test have been confirmed with extraordinarily high precision.

Some observations, however, and in contradiction to the SM. In particular, the discovery of neutrino oscillations shows that neutrinos are massive \cite{Nustatus}, which is unambiguous evidence for physics beyond the Standard Model. However, an extension to the SM\hairspace---\hairspace{}the seesaw mechanism\hairspace---\hairspace{}aims to account for both the neutrino masses and their smallness (six or more orders of magnitude smaller than that of the electron) by extending the model by new heavy particles coupling both to leptons and to Higgs doublets.

We pursue a broad search for the type-III seesaw signal \cite{SeesawIII:a} by examining the final state with at least three isolated prompt leptons ($e$, $\mu$) using proton--proton collision data collected by the CMS detector \cite{Chatrchyan:2008zzk} at the LHC in 2015.\footnote{This analysis has been approved by CERN for publication. A summary is thus also avaiable as a Physics Analysis Summary (PAS) \cite{CMS-PAS-EXO-16-002} on the CERN document server. The thesis author is also the PAS author.}

The most notable backgrounds are \WZ decaying to three leptons, fully leptonic \ttbar decays with a misidentified\footnote{The term ``misidentified'' may refer both to real leptons that arise from non-prompt decays, for instance: of hadrons, and to non-leptonic objects that are reconstructed as leptons.} lepton from a b-jet, leptonic \Z decays accompanied by a misidentified lepton, and leptonic \ZZ decays. In addition to these, there are rare backgrounds such as $\ttbar\Z$, $\ttbar\PW$, triboson, and Higgs production.
The $VV$ backgrounds ($V = \PW, \Z$) are generally well modeled by Monte Carlo (MC) simulation. Backgrounds with misidentified leptons, however, are not as easily modeled by MC simulation and are thus estimated from data. The background estimation methods employed in this search are enhanced versions of similar methods that have been used extensively in various CMS Run-I publications, e.\,g. \cite{Chatrchyan:2013xsw,Chatrchyan:2014aea,Khachatryan:2014mma,Khachatryan:2014jya}.

Prior results for this model include an 8\,\TeV CMS result \cite{CMS-PAS-EXO-14-001} which sets exclusion limits for the democratic scenario at $m_\Sigma = 250\,\GeV$ (expected) and $m_\Sigma = 278\,\GeV$ (observed) based on trilepton channels, and an 8\,\TeV ATLAS result in the $\ell\ell jj$ final state \cite{ATLAS-CERN-PH-EP-2015-094} which extends to higher mass values, but cannot be directly compared because of different choices of mixing parameters and other model constraints. Both these results use datasets with an integrated luminosity of 20\fbinv, whereas an older CMS result uses a 7\,\TeV dataset with 4.9\fbinv \cite{CMS-PAPER-EXO-11-073}.

Going from 8\,\TeV to 13\,\TeV, the signal cross section has increased by a factor of 3 for masses at the sensitivity limit between 300 and 400\,\GeV. Still, due to various analysis improvements which include new decay modes involving the Higgs boson, 4-lepton channels, new kinematic binning, and refined background methods, the sensitivity with the current \fullLumi dataset at 13\,\TeV exceeds that of the Run I analysis.

%The central feature of the CMS apparatus is a superconducting solenoid of 6\unit{m} internal diameter, providing a magnetic field of 3.8\unit{T}. Within the superconducting solenoid volume are a silicon pixel and strip tracker, a lead tungstate crystal electromagnetic calorimeter (ECAL), and a brass and scintillator hadron calorimeter (HCAL), each composed of a barrel and two endcap sections. Forward calorimeters extend the pseudorapidity~\cite{Chatrchyan:2008zzk} coverage provided by the barrel and endcap detectors. Muons are measured in gas-ionization detectors embedded in the steel flux-return yoke outside the solenoid. A more detailed description of the CMS detector, together with a definition of the coordinate system used and the relevant kinematic variables, can be found in Ref.~\cite{Chatrchyan:2008zzk}.

Before describing the details of the analysis, an overview of the Standard Model and its shortcomings will be presented, accompanied by a description of the phenomenology of the type-III seesaw model which is suitable to solve some of these issues (Chapter~\ref{chap:Theory}). A brief description of the structure and the most important features of both the LHC accelerator and the CMS detector that was used to collect the data for this search follows (Chapter~\ref{chap:Detector}). After these rather generic sections, Chapters~\ref{chap:Samples}--\ref{chap:Results} continue to describe the analysis itself as well as the results.
