\chapter{Triggers and Object Selection}
\label{sec:ObjectID}

The data for this search are collected using several dilepton triggers. The double electron trigger requires two electrons with \pt thresholds of 17\,\GeV on the leading electron and 12\,\GeV on the sub-leading electron. The double muon trigger requires two muons with \pt thresholds of 17 and 8\,\GeV on the leading and sub-leading muons, respectively. We use two muon/electron cross triggers, one of which requires a 17\,\GeV muon and a 12\,\GeV electron, while the other requires a 17\,\GeV electron and a 8\,\GeV muon.

Events that pass the trigger are required to satisfy additional selection criteria. Electrons with $\pt \geq 7\,\GeV$ and $|\eta| \leq 2.5$ as well as muons with $\pt \geq 5\,\GeV$ and $|\eta| \leq 2.4$ are reconstructed using the particle-flow (PF) algorithm which utilizes measured quantities from the tracker, calorimeter, and muon system \cite{CMS-PAS-PFT-09-001}. The matching candidate tracks must satisfy quality requirements and spatially match with the energy deposits in the ECAL and the tracks in the muon detectors, as appropriate.

Sources of background leptons include genuine leptons occurring inside or near jets, hadrons that punch through into the muon system and are misidentified as muons, hadronic showers with large electromagnetic fractions, or photon conversions. An isolation requirement -- imposing a selection based on the size of the lepton transverse momentum in comparison to the transverse momenta of other particles in its immediate neighborhood -- strongly reduces the background from misidentified leptons, since most of them occur inside jets. In this search, we use the multi-isolation variable which features a \pt-dependent isolation cone and pile-up correction \cite{CMS-PAS-SUS-15-008}. For electrons, the medium working point is employed, while for muons, we use the tight working point. Further details on the input variables, working points, and their efficiencies can be found in \cite{CMS-PAS-SUS-15-008}.

The signal leptons originate from the interaction point. After the isolation selection, the most significant background sources are residual non-prompt leptons from heavy quark decays, where the lepton tends to be more isolated because of the high \pt with respect to the jet axis. This background is reduced by requiring that the leptons satisfy $d_\textrm{z} \leq 0.1\,\cm$ where $d_\textrm{z}$ is the longitudinal impact parameter with respect to the primary interaction vertex, and that the impact parameter $d_\textrm{xy}$ between the track and the event vertex in the plane transverse to the beam axis be small: $d_\textrm{xy} \leq 0.05\,\cm$. The isolation and impact parameter criteria retain signal but significantly reject misidentified leptons.

The missing transverse momentum is calculated as the negative vectorial sum of the transverse momenta of all the PF candidates. The missing transverse energy \MET is defined as the magnitude of this vector. Jet energy corrections are applied to all jets and also propagated to the calculation of \MET \cite{CMS-PAS-JME-12-002}. We apply additional smearing to simulation samples to model the \MET resolutions we find in data as a function of jet activity and number of interaction vertices in an event.
