\chapter{Selection}
\label{chap:Selection}

\section{Object Selection}
\label{sec:Selection/Object}

\subsection{Electrons and Muons}
Events that pass the trigger are required to satisfy additional selection criteria. Electrons with $\pt \geq 7\GeV$ and $|\eta| \leq 2.5$ as well as muons with $\pt \geq 5\GeV$ and $|\eta| \leq 2.4$ are reconstructed using the particle-flow (PF) algorithm which utilizes measured quantities from the tracker, calorimeter, and muon system \cite{CMS-PAS-PFT-09-001}. The matching candidate tracks must satisfy quality requirements and spatially match with the energy deposits in the ECAL and the tracks in the muon detectors, as appropriate.

Sources of background leptons include genuine leptons occurring inside or near jets, hadrons that punch through into the muon system and are misidentified as muons, hadronic showers with large electromagnetic fractions, or photon conversions. Since the leptons from the seesaw signal are generally not related to hadronic activity, electrons and muons are expected to be spatially isolated from any jets occurring in the event. Furthermore, as jets occasionally produce leptons in their vincinity, an isolation requirement strongly reduces the background from such misidentified leptons.

The isolation requirement imposes a selection based on the size of the lepton transverse momentum in comparison to the transverse momenta of other particles in its immediate neighborhood. A common isolation discriminator is the the ``relative isolation'',
\begin{equation}
	I_\text{rel}(\Delta R^\textrm{max}) = \frac{\sum_\text{other}^{\Delta R < \Delta R^\textrm{max}} \pt^\text{other}}{\pt^\ell},
\end{equation}
defined as the transverse momentum sum of charged hadrons, neutral hadrons, and photons within a $\Delta R^\textrm{max}$ cone around the lepton candidate, divided by the lepton's own \pt.

However, as the masses of the heavy fermions in the seesaw model are high, their decay products are expected to be boosted and thus harder to distinguish geometrically. The relative isolation variable therefore does not provide good enough separation. We therefore use the multi-isolation discriminator which has better discrimination power in the case of boosted topologies \cite{CMS-PAS-SUS-15-008}. This is achieved using the following three input variables:

\begin{enumerate}
	\item A modified version of the relative isolation $I_\textrm{rel}$, called mini-isolation (\miniIso) \cite{Rehermann:2010vq}. Just like $I_\textrm{rel}$, it is the ratio of the scalar sum of the transverse momenta of charged hadrons, neutral hadrons, and photons within a cone around the lepton candidate; however, the cone radius depends on the transverse momentum of the lepton candidate itself as
		\begin{equation}
			\Delta R^\textrm{max}(\pt^\ell) = \dfrac{10\GeV}{\min\left[\max\left(\pt^\ell, 50\GeV\right), 200\GeV\right]}.
		\end{equation}
		The cone size thus varies between 0.2 and 0.05 as a function of the lepton \pt and is smaller for higher values of the lepton \pt. Hence, on the one hand, we reduce the chance of overlap with other objects as the lepton becomes stiffer. On the other hand, as misidentified leptons usually have rather low \pt, the larger cone size for low-\pt leptons improves the background rejection efficiency.
		
	\item We use the ratio of the lepton \pt and the \pt of a jet in which the lepton is contained:
		\begin{equation}
			\ptRatio = \dfrac{\pt^\ell}{\pt^\text{jet}}.
		\end{equation}
		Note that every lepton is, by definition, contained in a jet (which may contain nothing else besides the lepton). The \ptRatio variable thus tells us the relative share of jet momentum that is associated with the lepton. The larger this fraction, the lesser is the chance that the lepton stems from a hadronic decay or is misidentified. This variable is similiar to $I_\textrm{rel}$, except that the cone is replaced by the jet.
		
	\item To avoid rejecting leptons that fail the \ptRatio requirement because they overlap accidentally with another jet in the event, we consider the lepton candidate \pt along the axis of the residual momentum of the closest jet after subtracing the lepton \pt:
		\begin{equation}
			\ptRel=\frac{(\vec{p}(\text{jet})-\vec{p}(\ell))\cdot \vec{p}(\ell) }{|\vec{p}(\text{jet})-\vec{p}(\ell)|}.
		\end{equation}
		If the \ptRatio condition is not met while the lepton still has substantial momentum along the residual momentum axis, the lepton is allowed to pass.
\end{enumerate}
A lepton is considered to be isolated if the following condition is met:
\begin{equation}
	\miniIso < I_1 \wedge ( \ptRatio > I_2 \vee \ptRel > I_3 )
\end{equation}

The values of $I_\text{i}$, $i = 1,2,3,$ depend on the lepton flavor. For electrons, the medium working point is employed, while for muons, we use the tight working point. As the chance of misidentification is higher for electrons, tighter isolation values are used in this case (see Table~\ref{tab:isoWPs}). Further details on the input variables, these and other working points, and their efficiencies can be found in \cite{CMS-PAS-SUS-15-008}.

\begin{table}
\centering
\caption{Multi-isolation working points used in the analysis.} \label{tab:isoWPs}
\begin{tabular}{l ccc}
\hline\hline
Isolation value & Muons & Electrons  \\
\hline
$I_1$ & 0.20 & 0.16 \\
$I_2$ & 0.69 & 0.76 \\
$I_3$ [GeV] & 6.0 & 7.2 \\
\hline
\end{tabular}
\end{table}

The signal leptons originate from the interaction point. After the isolation selection, the most significant background sources are residual non-prompt leptons from heavy quark decays, where the lepton tends to be more isolated because of the high \pt with respect to the jet axis. This background is reduced by requiring that the leptons satisfy $d_\textrm{z} \leq 0.1\,\cm$ where $d_\textrm{z}$ is the longitudinal impact parameter with respect to the primary interaction vertex, and that the impact parameter $d_\textrm{xy}$ between the track and the event vertex in the plane transverse to the beam axis be small: $d_\textrm{xy} \leq 0.05\,\cm$. The isolation and impact parameter criteria retain signal but significantly reject misidentified leptons.

\subsection{Missing Transverse Momentum (\MET)}
The missing transverse momentum is calculated as the negative vectorial sum of the transverse momenta of all the PF candidates. The missing transverse energy \MET is defined as the magnitude of this vector. Jet energy corrections are applied to all jets and also propagated to the calculation of \MET \cite{CMS-PAS-JME-12-002}. We apply additional smearing to simulation samples to model the \MET resolutions we find in data as a function of jet activity and the number of interaction vertices in an event.


\section{Event Selection}
\label{sec:Selection}

For the three leading leptons, we apply offline thresholds of 20, 15, 10\GeV. We find that with these thresholds, the trigger efficiency of trilepton events is close to 100\%.

Events with an opposite-sign lepton pair with mass below 12 GeV are vetoed to reduce background from low-mass resonances.
%Furthermore, we reject trilepton events with an OSSF pair below the \Z boson mass window when the trilepton mass is within the \Z mass window. This cuts away background from asymmetric photon conversions in $\Z \to \ell\ell^* \to \ell\ell\gamma$, where the photon converts into two additional leptons, one of which is lost.
