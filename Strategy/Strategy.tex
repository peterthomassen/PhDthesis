\chapter{Search Strategy}
\label{sec:Strategy}

Candidate events in this search must have a total of at least three leptons, each of which can be either an electron or a muon. We classify multilepton events into search channels on the basis of the number of leptons, lepton flavor, lepton relative charges, charge and flavor combinations, and other kinematic quantities described below.

We classify each event in terms of the maximum number of opposite-sign same-flavor (OSSF) dilepton pairs that can be made by using each lepton only once. For example, both $\mu^+\mu^-\mu^-$ and $\mu^+\mu^-e^-$ are OSSF1, $\mu^+\mu^+e^-$ is OSSF0, and $\mu^+\mu^-e^+e^-$ is OSSF2. We denote a lepton pair of different flavors as $\ell\ell^\prime$.

We classify events as containing a leptonically-decaying \Z if at least one OSSF pair has $m_{\ell^+\ell^-}$ in the \Z mass window $91 \pm 10\,\GeV$. For $m_{\ell^+\ell^-}$ outside the \Z boson mass window, events are separated into bins below and above the \Z mass window. In cases of ambiguity (such as $\mu^+\mu^-\mu^-$), the pair below the \Z mass window takes precedence (thus shifting events from high mass to low mass, for a more separative background categorization). We refer to these three mass ranges as ``on-\Z'', ``below-\Z'', and ``above-\Z''.

The most important multilepton background processes are \WZ, \Z or \ttbar events in which there is a misidentified lepton, and \ZZ production. In addition, there are various rare background processes like WWZ or $\ttbar\PW$. However, the level of SM background varies considerably across channels; for example, channels containing OSSF pairs suffer from larger backgrounds than do channels with OSSF0. Hence, all these charge combinations are considered as different channels.

%Backgrounds can be tamed by binning in appropriate quantities. Since the signal leptons have relatively high \pt and since in some decay modes there is missing transverse energy (\MET) from accompanying neutrinos, we find that binning in $L_\textrm{T} + \MET$, where $L_\textrm{T}$ is the scalar sum of the lepton \pt's, separates the signal from the background. We find that lepton \pt binning alone gives about 20\,\% worse signal-to-background ratios in the most sensitive signal regions.
%Additional separation is achieved through the on- and off-\Z binning described above.
