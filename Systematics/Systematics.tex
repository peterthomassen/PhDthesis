\chapter{Systematic Uncertainties}
\label{chap:Systematics}

Since most of the signal regions are limited by statistics, systematic uncertainties play a minor role. The only regions where we expect 10 or more events are the signal regions with 3 leptons including an OSSF pair on or above-\Z, and $L_\textrm{T} + \MET < 550\,\GeV$. In these regions, the \WZ and \ttbar background uncertainties become relevant. However, channels with higher $L_\textrm{T} + \MET$ are more sensitive to the signal. The full list of uncertainties is found in Table~\ref{tab:Systematics}, along with their impact on a representative set of three of the most sensitive channels.

\begin{table}
\centering
\small
\caption{Systematic uncertainties. The channels listed here have three leptons and $550\,\GeV < L_\textrm{T} + \MET < 750\,\GeV$.} \label{tab:Systematics}
\begin{tabular}{l c c c c}
\hline\hline
 & & \multicolumn{3}{c}{Impact on background/signal estimate in channel with} \\
Source of uncertainty & Magnitude & no OSSF pair & OSSF pair above-\Z & OSSF pair on-\Z \\
\hline
\WZ normalization                & 50\,\%       & 13\,\%  & 2.8\,\% & 41\,\%  \\
\ZZ normalization                & 16\,\%       & 0.1\,\% & 0.5\,\% & 0.4\,\% \\
Integrated luminosity            & 2.7\,\%      & 0.6\,\% & 0.2\,\% & 0.3\,\% \\
Lepton ID and isolation          &  3\,\%       & 3\,\%   & 3\,\%   & 3\,\%   \\
\MET resolution/smearing         & 50\,\%       & 4.1\,\% & 6.3\,\% & 0.6\,\% \\
Pile-up reweighting              & 5\,\%        & 1.5\,\% & 0.3\,\% & 1.3\,\% \\
\ttbar misidentification rate    & 50\,\%       & 21\,\%  & 11\,\%  & 1.8\,\% \\
\Z + fake background             & 14\,\%       & 9.2\,\% & 1.1\,\% & 1.0\,\% \\
Rare MC cross section            & 50\,\%       & 11\,\%  & 2.7\,\% & 5.2\,\% \\
\\
Signal cross section             & 10\,\%       & 10\,\%  & 10\,\%  & 10\,\% \\
\\
\multicolumn{2}{l}{Total Background (for comparison)} & 0.3 events & 3.0 events & 3.5 events \\
\multicolumn{2}{l}{Signal ($m_\Sigma = 420\,\GeV$, for comparison)} & 0.8 events & 1.8 events & 0.8 events \\
\hline
\end{tabular}
\end{table}

The \ZZ and \ttbar uncertainties are based on the statistical uncertainties of the normalization regions; cross section uncertainties are thus not applied. For \WZ, we apply a 50\,\% uncertainty to account for the variation of the normalization factor depending on the \MET range chosen for normalization (see Sec.~\ref{sec:bkg_WZ}). For rare background processes, we apply a 50\,\% theory systematic uncertainty to cover both PDF as well as renormalization and factorization scale uncertainties. In the case of the signal, these uncertainties are covered by a 10\,\% systematic uncertainty \cite{CMS-PAS-EXO-14-001}.

For the \MET smearing procedure, a conservative uncertainty is determined by varying the amount of smearing by 50\,\%. Pile-up weights are evaluated by varying the minimzm-bias cross section by 5\,\% and propagating the impact on the pile-up weights through the analysis chain.

In general, systematic uncertainties are found by weighing events up or down or smearing them, then propagating those changes into the various bins of the analysis. The change in the expected backgrounds or signal yields in each bin corresponds to a systematic uncertainty, where we keep track of the relative sign of changes between different bins in order to take correlations and anti-correlations into account. Examples:
\begin{itemize}
	\item The luminosity uncertainty is correlated amongst all samples to which it is applied (i.\,e. MC samples that are not normalized to data). 
	\item As we apply the \MET smearing, events can migrate between $L_\textrm{T} + \MET$ bins. The uncertainty of the correction is thus anti-correlated between those bins. 
	\item An increase of the \WZ normalization by $1\sigma$ leads to a decrease of the measured \Z + jets fake rate, as we subtract \WZ background. Similarly, a $1\sigma$ increase of the \Z + jets fake rate leads to a decrease in the \WZ normalization, as the two control regions cannot be completely isolated from each other and have a (small) overlap. In all these cases, we take the relative signs of the changes into account to keep track of the correlations and anti-correlations.
\end{itemize}

Details on the fake rate uncertainties can be found in Sec.~\ref{sec:bkg_fakeLight}.
