\chapter{The Standard Model and The Seesaw Mechanism}

\section{Standard Model}
The Standard Model (SM) is a relativistic quantum field theory describing all known fundamental interactions between elementary particles with the exception of gravity, i.\,e. it describes electromagnetism as well as the weak and strong interactions. One has not yet succeeded integrating gravity into the same framework. However, since gravitational effects are negligible in the LHC energy range, gravity can safely be ignored for our purposes.

The SM makes use of several types of fields, each describing a different kind of particle. The model contains half-integer and integer spin particles (in units of $\hbar$) which are called fermions and bosons, respectively. The following sections are based on Refs.~\cite{i2003gauge,aitchison2003gauge} and elaborate on the individual types of particles in greater detail.

\begin{table}
	\centering
	\begin{tabular}{c||c|c|c|c}
		 & \textbf{particle} & $\begin{matrix}\textbf{mass} \\ \text{[MeV/$c^2$]}\end{matrix}$ & \textbf{spin} & $\begin{matrix}\textbf{electrical} \\ \textbf{charge} \text{ [$e$]}\end{matrix}$ \tabularnewline
		\hline 
		\hline 
		 & \multicolumn{4}{c}{} \tabularnewline
		 & \multicolumn{4}{c}{\textbf{fermions}} \tabularnewline
		\hline
		\multirow{6}{*}{$\begin{matrix}\textbf{leptons} \\ \\ L = 1, \\ B = 0\end{matrix}$} & $e$ & 0.511 & \nicefrac{1}{2} & $-1$ \tabularnewline
		 & $\nu_{e}$ & $0 < m_{\nu_e} < 2.2\cdot10^{-6}$ & \nicefrac{1}{2} & 0 \tabularnewline
		 & $\mu$ & 105.7 & \nicefrac{1}{2} & $-1$ \tabularnewline
		 & $\nu_{\mu}$ & $0 < m_{\nu_\mu} < 0.17$ & \nicefrac{1}{2} & 0 \tabularnewline
		 & $\tau$ & $1.78\cdot 10^3$ & \nicefrac{1}{2} & $-1$ \tabularnewline
		 & $\nu_{\tau}$ & $0 < m_{\nu_\tau} < 15.5\cdot10^{-6}$ & \nicefrac{1}{2} & 0 \tabularnewline
		\hline
		\multirow{6}{*}{$\begin{matrix}\textbf{quarks} \\ \\ L = 0, \\ B = \nicefrac{1}{3}\end{matrix}$} & $u$ & 2.4 & \nicefrac{1}{2} & $\nicefrac{2}{3}$ \tabularnewline
		 & $d$ & 4.8 & \nicefrac{1}{2} & $-\nicefrac{1}{3}$ \tabularnewline
		 & $c$ & $1.27\cdot 10^3$ & \nicefrac{1}{2} & $\nicefrac{2}{3}$ \tabularnewline
		 & $s$ & 104 & \nicefrac{1}{2} & $-\nicefrac{1}{3}$ \tabularnewline
		 & $t$ & $171.2 \cdot 10^3$ & \nicefrac{1}{2} & $\nicefrac{2}{3}$ \tabularnewline
		 & $b$ & $4.2\cdot 10^3$ & \nicefrac{1}{2} & $-\nicefrac{1}{3}$ \tabularnewline
		\hline
		 & \multicolumn{4}{c}{} \tabularnewline
		 & \multicolumn{4}{c}{\textbf{bosons}} \tabularnewline
		\hline
		\multirow{5}{*}{$\begin{matrix}L = 0, \\ B = 0\end{matrix}$} & $\gamma$ & 0 & 1 & 0 \tabularnewline
		 & $g$ & 0 & 1 & 0 \tabularnewline
		 & $Z$ & $91.2\cdot 10^3$ & 1 & 0 \tabularnewline
		 & $W^\pm$ & $80.4\cdot 10^3$ & 1 & $\pm 1$ \tabularnewline
		 & $H$ & $(\text{125...127}) \cdot 10^3$ & 0 & 0 \tabularnewline
	\end{tabular}
	\caption{Elementary particles in the Standard Model \cite{PhysRevD.86.010001}. The Higgs boson $H$ has not yet been observed with sufficient significance, and its precise properties remain unknown \cite{Higgs}. For electrically charged particles, anti-particles with opposite charge exist. Neutrinos presumably have anti-particles with opposite chirality. Anti-particles have been omitted in this summary.}
	\label{tab:SM}
\end{table}

\subsection{Fermions}
The fermion group consists of two subgroups named leptons and quarks; both of them are subdivided into three so-called ``generations'', or ``flavors''.

\subsubsection*{Leptons}
The three lepton generations are
\begin{eqnarray}
	\begin{pmatrix}\nu_e \\ e \end{pmatrix}, \quad
	\begin{pmatrix}\nu_\mu \\ \mu \end{pmatrix}, \quad
	\begin{pmatrix}\nu_\tau \\ \tau \end{pmatrix},
\end{eqnarray}
where $e$, $\mu$, $\tau$ are similar particles of electrical charge $-1$ and spin \nicefrac{1}{2}. However, their masses are quite different (see Table~\ref{tab:SM}). In interactions, they usually appear with the corresponding neutrino $\nu_i$. In addition to these six particles, there are also six antiparticles with opposite charge sign and lepton number.\footnote{It is also possible that neutrinos are Majorana fermions and thus their own anti-particles. This question has not yet been answered experimentally.} The present analysis is mainly concerned with events exhibiting three or more electrons or muons.

\subsubsection*{Quarks}
There are six quarks called up, down, charm, strange, top, and bottom quark. They are organized in generations as follows:
\begin{eqnarray}
	\begin{pmatrix}u \\ d \end{pmatrix}, \quad
	\begin{pmatrix}c \\ s \end{pmatrix}, \quad
	\begin{pmatrix}t \\ b \end{pmatrix},
\end{eqnarray}
where the particles in the upper row are of electrical charge $+\nicefrac{2}{3}$, and those in the lower row have electrical charge $-\nicefrac{1}{3}$. Anti-quarks have opposite charge and baryon number. As quarks are subject to strong interaction, they carry an additional ``color'' charge which is either ``red'', ``green'', or ``blue''.

Quarks have not been observed individually; instead, they form bound states such that the electrical charge is integer and the color charge vanishes or adds up to ``white'' (i.\,e. all three colors are present). Particles consisting of three quarks are called baryons (for example the proton: $p \equates uud$), quark-antiquark combinations are called mesons (for example the pion: $\pi^+ \equates u \bar d$).

\subsection{Bosons}
The quantum field theory on which the SM is built is invariant under Lorentz and CPT transformations, and certain gauge transformations. To prevent the theory from losing this invariance, the existence of so-called gauge bosons was predicted and observed. In addition, these particles act as the force carriers of the fundamental forces.

The most well-known one is the massless photon ($\gamma$) which is electrically neutral and mediates the electromagnetic interaction. A very similar particle, although massive, is the $Z$ boson which can interact electromagnetically and weakly. Furthermore, the charged $W^+$ and $W^-$ bosons exist.\footnote{In fact, the $\gamma$ and $Z$ fields are superpositions of the more fundamental $B$ and $W^0$ fields. The $B$ field arises from spontaneous $U(1)$ symmetry breaking, while the $W^i$ come from the breaking of $SU(2)$.} Conceptually, they have the same origin as the $Z$ boson, which is why they take part in the same interactions. A great theoretical achievement was the unification of the electromagnetic and the weak interaction into a combined concept, the electroweak interaction.

The strong force between quarks is carried by the massless gluons ($g$) which come in eight different color-anticolor combinations.

\section{Extension of the Standard Model}
While the Standard Model predicts the electromagnetic, weak, and strong phenomena with extraordinary precision, there are open questions that are not addressed by the SM:
\begin{itemize}
	\item The Standard Model does not account for \textit{gravity} at all. It is described by General Relativity, and it is believed that, in principle, a unification of the theories is possible.
	\item The Standard Model contains a number of parameters that differ from expectation by several orders of magnitude for unknown reasons. For example, the mass of the Higgs boson was expected to be around $10^{15}\GeV$ due to top quark loops, but now it appears to be on the electroweak scale, and it seems that there are delicate cancellations from other fields. This issue is referred to as the \textit{Hierarchy Problem}.
	\item The Standard Model does not explain \textit{Dark Matter}.
\end{itemize}

Several attempts have been made to find remedies for these issues from a theoretical point of view, and because they come with predictions of new particles, they are subject to experimental examination.

\section{Seesaw Mechanism}
