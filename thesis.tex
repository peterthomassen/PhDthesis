\documentclass[11pt]{ruthesis}


% The page number in Rutgers thesis format is so high up in the
% page that normal printers can not print it out! The following is to
% lower page numbers down a bit, so we can see them in a draft
% version to the committee members. Do NOT use this change in the
% FINAL version to be submitted to the Graduate School.
%\addtolength{\topmargin}{0.3cm}


\usepackage[utf8]{inputenc}	% UTF-8
\usepackage[T1]{fontenc}	% Improve umlaut support
\usepackage{textcomp}		% https://tex.stackexchange.com/questions/49610/use-biblatex-and-utf8
\usepackage{ae}				% German Sharp S
\usepackage{lmodern}
\usepackage[pdftex,bookmarks=true,bookmarksopenlevel=1,colorlinks=true,pdfborder={0 0 0}]{hyperref}
\usepackage{microtype}
\usepackage{float}
\usepackage{rotating}
\usepackage{comment} 
\usepackage{graphicx}
%\usepackage[numbers]{natbib}
\usepackage{slashed}
\usepackage{lmodern}
\usepackage{mathrsfs}
\usepackage{nextpage}
\usepackage{setspace}
\usepackage{amssymb}
\usepackage{xspace}
\usepackage{ifthen}
\usepackage{url}
\usepackage{multirow}
\usepackage{booktabs}
\usepackage{amsmath}
\usepackage{savesym}
\usepackage{braket}
\usepackage{subcaption}
\usepackage{units}
\usepackage[backend=bibtex,sorting=none]{biblatex}
\bibliography{thesis.bib}

\newcommand{\equates}{\mathrel{\widehat{=}}}
\providecommand{\fixme}[1]{{\sffamily{\bfseries{}FIXME:} #1}}
\newcommand{\ptRatio}{\ensuremath{\pt^\text{ratio}}\xspace}
\newcommand{\ptRel}{\ensuremath{\pt^\text{rel}}\xspace}
\newcommand{\relIso}{\ensuremath{I_\text{rel}}\xspace}
\newcommand{\miniIso}{\ensuremath{I_\text{mini}}\xspace}
\newcommand{\multiIso}{\ensuremath{I_\text{multi}}\xspace}
\newcommand\hairspace{\kern .08333em }

\pdfpkresolution=2400 % needed for \dlsh and \drsh

\special{papersize=8.5in,11in} %***for A4-default configurations on servers

% Include defined symbols
\input{pennames-pazo.sty}
\input{ptdr-definitions.sty}
%%%%%%%%%%%%%%%%%%%%%%%%%%%%%%%%%%%%%%%%%%%%%%%%%%%%%%%%%%%%%%%%%%%%
%
%  Thesis definitions style file
%
%  Authors: Emmanuel Conreras-Campana and Christian Contreras-Campana
%
%  Last Modification: 4 Aug 2014
%
%%%%%%%%%%%%%%%%%%%%%%%%%%%%%%%%%%%%%%%%%%%%%%%%%%%%%%%%%%%%%%%%%%%%

%
% Put additional commands, abbreviations etc. here
%
\providecommand{\fullLumi}{2.3$\fbinv$\xspace}

% Kinematic quantities
\newcommand{\MT}{\ensuremath{M_{\mathrm{T}}}\xspace}
\newcommand{\ST}{\ensuremath{S_{\mathrm{T}}}\xspace}
\newcommand{\ETslashvec}{\ensuremath{\vec{E}_{\mathrm{T}}\hspace{-1.1em}/\kern0.45em}\xspace}
\newcommand{\ptveci}{\ensuremath{\vec{p}^{~i}_{\mathrm{T}}}\xspace}
\newcommand{\ptsq}{\ensuremath{p^{2}_{\mathrm{T}}}\xspace}
\newcommand{\ptTrack}{\ensuremath{p^{\textrm{track}}_{\mathrm{T}}}\xspace}
\newcommand{\ptTauh}{\ensuremath{p^{\tauh}_{\mathrm{T}}}\xspace}
\newcommand{\ptZboson}{\ensuremath{p^{\cPZ}_{\mathrm{T}}}\xspace}
\newcommand{\ptStrip}{\ensuremath{p^{\textrm{strip}}_{\mathrm{T}}}\xspace}
\newcommand{\kti}{\ensuremath{k_{\mathrm{T},i}}\xspace}
\newcommand{\ktj}{\ensuremath{k_{\mathrm{T},j}}\xspace}
\newcommand{\ktSqNi}{\ensuremath{k^{2n}_{\mathrm{T},i}}\xspace}
\newcommand{\ktSqNj}{\ensuremath{k^{2n}_{\mathrm{T},j}}\xspace}
\newcommand{\WZ}{\ensuremath{\PW\Z}\xspace}
\newcommand{\ZZ}{\ensuremath{\Z\Z}\xspace}


% Confidencel level
\providecommand{\CL}{CL\xspace}

% Hadronic tau
\providecommand{\tauh}{\ensuremath{\Pgt_\textrm{h}}\xspace}

% Sleptons
\providecommand{\slepton}{\ensuremath{\widetilde{\ell}}\xspace}
\providecommand{\selectron}{\ensuremath{\widetilde{\cmsSymbolFace{e}}}\xspace}
\providecommand{\smuon}{\ensuremath{\widetilde{\mu}}\xspace}
\providecommand{\stau}{\ensuremath{\widetilde{\tau}}\xspace}

% Squarks
\providecommand{\squark}{\ensuremath{\widetilde{\cmsSymbolFace{q}}}\xspace}
\providecommand{\sbone}{\ensuremath{\widetilde{\cmsSymbolFace{b}}_{1}}\xspace}
\providecommand{\stop}{\ensuremath{\widetilde{\cmsSymbolFace{t}}}\xspace}

% Neutralinos
\providecommand{\nall}{\ensuremath{\widetilde{\chi}^{0}}\xspace}
\providecommand{\nalli}{\ensuremath{\widetilde{\chi}^{0}_{i}}\xspace}
\providecommand{\none}{\ensuremath{\widetilde{\chi}^{0}_{1}}\xspace}
\providecommand{\ntwo}{\ensuremath{\widetilde{\chi}^{0}_{2}}\xspace}
\providecommand{\nthree}{\ensuremath{\widetilde{\chi}^{0}_{3}}\xspace}

% Charginos
\providecommand{\call}{\ensuremath{\widetilde{\chi}^{\pm}}\xspace}
\providecommand{\cone}{\ensuremath{\widetilde{\chi}^{\pm}_{1}}\xspace}
\providecommand{\conem}{\ensuremath{\widetilde{\chi}^{-}_{1}}\xspace}
\providecommand{\conep}{\ensuremath{\widetilde{\chi}^{+}_{1}}\xspace}
\providecommand{\ctwo}{\ensuremath{\widetilde{\chi}^{\pm}_{2}}\xspace}

% Gravitino or Goldstino
\providecommand{\goldstino}{\ensuremath{\widetilde{\cmsSymbolFace{G}}}\xspace}

% anti-stop
\renewcommand{\PASQt}{\ensuremath{\widetilde{\cmsSymbolFace{t}}^*}\xspace}

% Left down and right down angled arrows
\DeclareFontFamily{U}{mathb}{\hyphenchar\font45}
\DeclareFontShape{U}{mathb}{m}{n}{
      <5> <6> <7> <8> <9> <10> gen * mathb
      <10.95> mathb10 <12> <14.4> <17.28> <20.74> <24.88> mathb12
      }{}
\DeclareSymbolFont{mathb}{U}{mathb}{m}{n}
\DeclareFontSubstitution{U}{mathb}{m}{n}
\DeclareMathSymbol{\dlsh}                  {3}{mathb}{"EA}
\DeclareMathSymbol{\drsh}                  {3}{mathb}{"EB}



% Begin document
\begin{document}


% Begin title page

\phd 

\title{Search for Type-III Seesaw Heavy Fermions with Multilepton Final States using \fullLumi of \boldmath$\sqrt{\lowercase{s}} = 13\,\TeV$ proton--proton Collision Data}

% Your name must exactly match what is on your transcript
\author{Peter Thomassen}

\program{Physics and Astronomy}

\director{Dr. Sunil Somalwar}

% Physics has five committee members, including your advisor and one member external to the department
% Side note: your external member must be approved by the GSNB - send the dept. grad director a copy of their CV to do that
\approvals{5}

% This is the month you will be awarded the degree, so one of January, May, or October
\submissionmonth{May} 

\submissionyear{2016}

% Begin beforepreface

\copyrightpage % ruthesis.cls boolean to turn copyright on and must be called before \beforepreface is called

% Page ii is the abstract - Must not exceed 2,450 characters and 350 words.
% The abstract should be relatively jargon-free but still contain any relevant keywords for search engines.
\abstract{
A search for type-III seesaw signal in events with three or more electrons or muons is presented. The data sample corresponds to \fullLumi\ of integrated luminosity in $pp$ collisions at $\sqrt{s} = 13\,\TeV$ collected by the CMS experiment at the LHC. Since the signal populates channels with at least three leptons and diverse kinematic properties, the data is binned in exclusive channels. The primary selection is based on the number of leptons and the invariant mass of opposite-sign dilepton systems which helps discriminate the signal against the Standard Model background. The final optimization for the type-III seesaw signal is based on the sum of leptonic transerve momenta and missing transverse energy. Control samples in data are used to check the robustness of background evaluation techniques and to minimize the reliance on simulation. The observations are consistent with expectations from Standard Model processes. The results are used to exclude heavy fermions of the type-III seesaw model with masses below 430\,\GeV.}


\beforepreface


% Begin afterpreface

\acknowledgements{
Many individuals and institutions have supported me during my Ph.\,D. studies and thesis writing. I would like to express my graditude especially to my supervisor Prof.~Sunil Somalwar whose advice and guidance were extremely valuable, and to the members of our research groups, especially Dr.~Richard Gray, Dr.~Matt Walker, and Dr.~Halil Saka, as well as to my fellow graduate students Sanjay Arora, Christian Contreras-Campaña, Emmanuel Contreras-Campaña, Maximilian Heindl, Shruti Panwalkar, Patrick Zywicki for their helpful discussions, input, and distractions. Special thanks go to Prof.~Scott Thomas for his guidance on theoretical matters, and to Ezio Torassa and Andrea Gozzelino for the collaboration on the seesaw model, and to my dear fellow student Elliot ``Tote'' Hughes.

My involvement with Rutgers University started initially with an exchange organized by Würzburg University. Financial support for the exchange was provided by the German Academic Exchange Service (DAAD). Further funds, for travel purposes as well as generally, came from several scholarships of the Foundation of German Business (Stiftung der Deutschen Wirtschaft, sdw), COLA support from the National Science Foundation (NSF), as well as from Rutgers University and Prof.~Somalwar's NSF grant.

This thesis contains material from previous publications of the author \cite{Thomassen2012,CMS-PAS-EXO-16-002}.
}


%\dedication{
%\begin{center}
%\begin{flushleft}
%\end{flushleft}
%\em{
%I dedicate this dissertation to Emmanuel to whom
%this thesis would not have been possible.}
%\end{center}
%}


\tablespage % ruthesis.cls boolean to turn list of tables on and must be called before \afterpreface is called
\figurespage % ruthesis.cls boolean to turn list of tables on and must be called before \afterpreface is called

\afterpreface

\chapter{Introduction}
\label{sec:Introduction}

The discovery of neutrino oscillations shows that neutrinos are massive \cite{Nustatus}, which is unambiguous evidence for physics beyond the Standard Model (SM). Many extensions of the SM have been proposed so far, among which the seesaw mechanism is an appealing possibility \cite{SeesawI,typeIa,typeIb,typeIe,typeIIa,typeIIb,typeIIc,typeIId,typeIIe,SeesawIII:a,Seesawinverse}. The seesaw mechanism introduces new heavy particles coupling both to leptons and to Higgs doublets, and accounts for both the neutrino masses and their smallness (six or more orders of magnitude smaller than that of the electron).

%These new heavy particles could be weak-singlet fermions (type I \cite{SeesawI,typeIa,typeIb,typeIe}); weak-triplet scalars (type II \cite{typeIIa,typeIIb,typeIIc,typeIId,typeIIe}); or weak-triplet fermions (type III \cite{SeesawIII:a}). They generate a small Majorana mass for the neutrinos, given by, in type I and type III models: $m_{\nu}=Y^T~ \mathcal{M}^{-1}~ Y ~ v^2/2$, where $Y^T$ is the transpose of the Yukawa coupling matrix between the mediators and the SM fermions, $\mathcal{M}$ is the mass matrix of the heavy partner of the neutrinos and $~v$ is the Higgs vacuum expectation value. When mediator mass is large enough (of order of $10^{14}\,\GeV$), small neutrino masses are generated even for order $O(1)$ Yukawa couplings. If $M$ is smaller (of the order of a few hundreds of \GeV, as in LHC searches), either smaller Yukawa couplings (i.\,e. ``natural'' mixing angles of the order of $10^{-6}$) are required to generate small neutrino masses, or an alternative suppression mechanism is needed (e.\,g. seesaw inverse mechanism \cite{Seesawinverse}). The possible allowed values of the mixing parameters and their products are constrained by precision electroweak data \cite{typeIIIconstraint}.

Within the type-III seesaw model \cite{SeesawIII:a}, the neutrino is considered a Majorana particle whose mass arises via the mediation of massive fermion partners. These massive partners are the fermionic $SU(2)$ triplet of the heavy Dirac charged leptons $\Sigma^\pm$, and the heavy Majorana neutral lepton $\Sigma^0$, coupling both to the leptons and to the Higgs doublets. During proton-proton collisions, the heavy fermion particles may be pair-produced through electroweak interactions in both charged-charged and charged-neutral pairs as can be seen in Fig.~\ref{fig:SeesawProduction}.

\begin{figure}[b]
\begin{center}
	\includegraphics[width=.5\textwidth]{Introduction/SeesawProduction-SpSm}%
	\includegraphics[width=.5\textwidth]{Introduction/SeesawProduction-SpmS0}
	\caption{Examples of Feynman diagrams for heavy fermion production in the type-III seesaw model.
	\label{fig:SeesawProduction}}
\end{center}
\end{figure}

We conduct a search for this signal by examining the final state with at least three electrons or muons. The primary decay channels of interest are $\Sigma^\pm \to \PW^\pm \nu$, $\Sigma^\pm \to \Z \ell^\pm$, $\Sigma^\pm \to \PH \ell^\pm$, $\Sigma^0 \to \PW^\pm \ell^\mp$, $\Sigma^0 \to \Z \nu$, $\Sigma^0 \to \PH \nu$, where $\ell = e, \mu$. Decays of $\Sigma^0 \Sigma^\pm$ and $\Sigma^+ \Sigma^-$ pairs result in 27 different production processes and can naturally lead to multilepton final states if several \PW\ or \Z\ bosons are involved, either directly or via a Higgs boson decay. An example Feynman diagram for one of the most relevant processes with three leptons in the final state, $\Sigma^\pm \Sigma^0 \to \PW^\pm \nu \PW^\pm \ell^\mp$ with leptonic $\PW^\pm$ decays, is shown in Fig.~\ref{fig:SeesawDecay}.
The decay rate of a $\Sigma$ to a given lepton $\ell$ is proportional to $v_{\ell N} = \frac{V_\ell}{\sqrt{|V_e|^2 + |V_\mu|^2 + |V_\tau|^2}}$. In the democratic scenario, the mixing parameters $V_\ell$ are the same for all the leptons.
%Further details on the signal phenomenology and generation can be found in Sec.~\ref{sec:Samples/Signal}.

\begin{figure}[t]
\begin{center}
	\includegraphics[width=.5\textwidth]{Introduction/Seesaw}
	\caption{Feynman diagram example of the fermion production and decay in the type-III seesaw model.
	\label{fig:SeesawDecay}}
\end{center}
\end{figure}

Prior results for this model include the EXO-14-001 PAS \cite{CMS-PAS-EXO-14-001} which set exclusion limits for the democratic scenario at $m_\Sigma = 250\,\GeV$ (expected) and $m_\Sigma = 278\,\GeV$ (observed) based on trilepton channels, and an ATLAS result in the $\ell\ell jj$ final state \cite{ATLAS-CERN-PH-EP-2015-094} which extends to higher mass values, but cannot be directly compared because of different choices of mixing parameters and other model constraints. Both these results use 8\,\TeV datasets with an integrated luminosity of 20\fbinv, whereas an older CMS result uses a 7\,\TeV dataset with 4.9\fbinv \cite{CMS-PAPER-EXO-11-073}.

Going from 8\,\TeV to 13\,\TeV, the signal cross section has increased by a factor of 3 for masses at the sensitivity limit between 300 and 400\,\GeV. Still, due to various analysis improvements which include new decay modes involving the Higgs boson, 4-lepton channels, new kinematic binning, and refined background methods, the sensitivity with the current \fullLumi dataset at 13\,\TeV exceeds that of the Run I analysis.

We pursue a broad search in final states with at least three isolated prompt leptons ($e$, $\mu$). The most notable backgrounds are \WZ decaying to three leptons, fully leptonic \ttbar decays with a misidentified\footnote{Note that the term ``misidentified'' may refer both to real leptons that arise from non-prompt decays of hadrons, for instance, and to non-leptonic objects that are reconstructed as leptons.} lepton from a b-jet, leptonic \Z decays accompanied by a misidentified lepton usually from a jet, and leptonic \ZZ decays. In addition to these, there are rare backgrounds such as $\ttbar\Z$, $\ttbar\PW$, triboson, and Higgs production.

The $VV$ backgrounds are generally well modeled by Monte Carlo (MC) simulation ($V = \PW, \Z$). Backgrounds with misidentified leptons, however, are not as easily modeled by MC simulation and are thus estimated from data.

The background estimation methods employed in this search have been used extensively in various CMS Run-I publications, e.\,g. \cite{Chatrchyan:2013xsw,Chatrchyan:2014aea,Khachatryan:2014mma,Khachatryan:2014jya}.

The central feature of the CMS apparatus is a superconducting solenoid of 6\unit{m} internal diameter, providing a magnetic field of 3.8\unit{T}. Within the superconducting solenoid volume are a silicon pixel and strip tracker, a lead tungstate crystal electromagnetic calorimeter (ECAL), and a brass and scintillator hadron calorimeter (HCAL), each composed of a barrel and two endcap sections. Forward calorimeters extend the pseudorapidity~\cite{Chatrchyan:2008zzk} coverage provided by the barrel and endcap detectors. Muons are measured in gas-ionization detectors embedded in the steel flux-return yoke outside the solenoid. A more detailed description of the CMS detector, together with a definition of the coordinate system used and the relevant kinematic variables, can be found in Ref.~\cite{Chatrchyan:2008zzk}.

\chapter{Theoretical Overview}
\label{chap:Theory}

\section{Standard Model}
The Standard Model (SM) is a relativistic quantum field theory describing all known fundamental interactions between elementary particles with the exception of gravity, i.\,e. it describes electromagnetism as well as the weak and strong interactions. One has not yet succeeded integrating gravity into the same framework. However, since gravitational effects are negligible LHC energies, gravity can be safely ignored for our purposes.

The SM makes use of several types of fields, each describing a different kind of particle. The model contains half-integer and integer spin particles (in units of the reduced Planck constant $\hbar$) which are called fermions and bosons, respectively. Refs.~\cite{i2003gauge,aitchison2003gauge} elaborate on the individual types of particles in greater detail.\footnote{The present section as well as Sec.~\ref{sec:Theory/Shortcomings} are largely taken from Ref.~\cite{Thomassen2012} (the author's Master's Thesis).}

\begin{table}
	\centering
	\begin{tabular}{c||c|c|c|c}
		 & \textbf{particle} & $\begin{matrix}\textbf{mass} \\ \text{[MeV/$c^2$]}\end{matrix}$ & \textbf{spin} & $\begin{matrix}\textbf{electrical} \\ \textbf{charge} \text{ [$e$]}\end{matrix}$ \tabularnewline
		\hline 
		\hline 
		 & \multicolumn{4}{c}{} \tabularnewline
		 & \multicolumn{4}{c}{\textbf{fermions}} \tabularnewline
		\hline
		\multirow{6}{*}{$\begin{matrix}\textbf{leptons} \\ \\ L = 1, \\ B = 0\end{matrix}$} & $e$ & 0.511 & \nicefrac{1}{2} & $-1$ \tabularnewline
		 & $\nu_{e}$ & $0 < m_{\nu_e} < 2.2\cdot10^{-6}$ & \nicefrac{1}{2} & 0 \tabularnewline
		 & $\mu$ & 105.7 & \nicefrac{1}{2} & $-1$ \tabularnewline
		 & $\nu_{\mu}$ & $0 < m_{\nu_\mu} < 0.17$ & \nicefrac{1}{2} & 0 \tabularnewline
		 & $\tau$ & $1.78\cdot 10^3$ & \nicefrac{1}{2} & $-1$ \tabularnewline
		 & $\nu_{\tau}$ & $0 < m_{\nu_\tau} < 15.5\cdot10^{-6}$ & \nicefrac{1}{2} & 0 \tabularnewline
		\hline
		\multirow{6}{*}{$\begin{matrix}\textbf{quarks} \\ \\ L = 0, \\ B = \nicefrac{1}{3}\end{matrix}$} & $u$ & 2.4 & \nicefrac{1}{2} & $\nicefrac{2}{3}$ \tabularnewline
		 & $d$ & 4.8 & \nicefrac{1}{2} & $-\nicefrac{1}{3}$ \tabularnewline
		 & $c$ & $1.27\cdot 10^3$ & \nicefrac{1}{2} & $\nicefrac{2}{3}$ \tabularnewline
		 & $s$ & 104 & \nicefrac{1}{2} & $-\nicefrac{1}{3}$ \tabularnewline
		 & $t$ & $173.34 \cdot 10^3$ & \nicefrac{1}{2} & $\nicefrac{2}{3}$ \tabularnewline
		 & $b$ & $4.2\cdot 10^3$ & \nicefrac{1}{2} & $-\nicefrac{1}{3}$ \tabularnewline
		\hline
		 & \multicolumn{4}{c}{} \tabularnewline
		 & \multicolumn{4}{c}{\textbf{bosons}} \tabularnewline
		\hline
		\multirow{5}{*}{$\begin{matrix}L = 0, \\ B = 0\end{matrix}$} & $\gamma$ & 0 & 1 & 0 \tabularnewline
		 & $g$ & 0 & 1 & 0 \tabularnewline
		 & \Z & $91.2\cdot 10^3$ & 1 & 0 \tabularnewline
		 & $\PW^\pm$ & $80.4\cdot 10^3$ & 1 & $\pm 1$ \tabularnewline
		 & \PH & $125.09 \cdot 10^3$ & 0 & 0 \tabularnewline
	\end{tabular}
	\caption{Elementary particles in the Standard Model \cite{PhysRevD.86.010001,ATLAS:2014wva,Aad:2015zhl}. For electrically charged particles, anti-particles with opposite charge exist. Neutrinos presumably have anti-particles with opposite chirality. Anti-particles have been omitted in this summary.}
	\label{tab:SM}
\end{table}

\subsection{Fermions}
The fermion group\footnote{``Group'' is not meant in the mathematical sense here.} consists of two subgroups named leptons and quarks; both of them are subdivided into three so-called ``generations'', or ``flavors''.

\subsubsection*{Leptons}
The three lepton generations are
\begin{eqnarray}
	\begin{pmatrix}\nu_e \\ e \end{pmatrix}, \quad
	\begin{pmatrix}\nu_\mu \\ \mu \end{pmatrix}, \quad
	\begin{pmatrix}\nu_\tau \\ \tau \end{pmatrix},
\end{eqnarray}
where $e$, $\mu$, $\tau$ are similar particles of electrical charge $-1$ and spin \nicefrac{1}{2}. However, their masses are quite different (see Table~\ref{tab:SM}). In interactions, they usually appear with the corresponding neutrino $\nu_\ell$, $\ell = e, \mu, \tau$.

In addition to these six particles, there are also six antiparticles with opposite charge sign and lepton number.\footnote{It is also possible that neutrinos are Majorana fermions and thus their own anti-particles, as is the case in the type-III seesaw model, for example. This question has not yet been answered experimentally.} The present analysis is concerned with events exhibiting three or more electrons or muons.

\subsubsection*{Quarks}
There are six quarks called up, down, charm, strange, top, and bottom quark. They are organized in generations as follows:
\begin{eqnarray}
	\begin{pmatrix}u \\ d \end{pmatrix}, \quad
	\begin{pmatrix}c \\ s \end{pmatrix}, \quad
	\begin{pmatrix}t \\ b \end{pmatrix},
\end{eqnarray}
where the particles in the upper row are of electrical charge $+\nicefrac{2}{3}$, and those in the lower row have electrical charge $-\nicefrac{1}{3}$. Anti-quarks have opposite charge and baryon number. As quarks are subject to strong interaction, they carry an additional ``color'' charge which is either ``red'', ``green'', or ``blue''.

Quarks have not been observed individually. In the SM, they thus form bound states such that the electrical charge is integer and the color charge vanishes or adds up to ``white'' (i.\,e. all three colors are present). Particles consisting of three quarks are called baryons (for example the proton: $p \equates uud + \textrm{valence quarks}$), and quark--antiquark combinations are called mesons (for example the pion: $\pi^+ \equates u \bar d$).

\subsection{Bosons}
The quantum field theory on which the SM is built is invariant under Lorentz and CPT transformations, and under certain gauge transformations. To prevent the theory from losing this invariance, the existence of so-called gauge bosons was predicted, and indeed later observed. These particles act as the force carriers of the fundamental forces.

The most well-known one is the massless photon ($\gamma$) which is electrically neutral and mediates the electromagnetic interaction. A very similar particle, although massive, is the \Z boson which can interact electromagnetically and weakly. Furthermore, the charged $\PW^+$ and $\PW^-$ bosons exist. Conceptually, they have the same origin as the \Z boson, which is why they take part in the same interactions.\footnote{In fact, the $\gamma$ and \Z fields are superpositions of the more fundamental $B$ and $W^0$ fields. The $B$ field arises from spontaneous $U(1)$ symmetry breaking, while the $W^i$ come from the breaking of $SU(2)$.} A great theoretical achievement was the unification of the electromagnetic and the weak interaction into a combined concept, the electroweak interaction.

The strong force between quarks is carried by the massless gluons ($g$) which come in eight different color-anticolor combinations.

\section{Shortcomings of the Standard Model}
\label{sec:Theory/Shortcomings}

While the Standard Model predicts the electromagnetic, weak, and strong phenomena with extraordinary precision, there are open questions that are not addressed by the SM:
\begin{itemize}
	\item The Standard Model does not account for \textit{gravity} at all. It is described by General Relativity, and it is believed that, in principle, a unification of the theories is possible.
	\item The Standard Model contains a number of parameters that differ from expectation by several orders of magnitude for unknown reasons. For example, the mass of the Higgs boson was expected to be around $10^{15}\GeV$ due to top quark loops, but it is in fact on the electroweak scale \cite{Aad:2015zhl}. This issue is referred to as the \textit{Hierarchy Problem} \cite{Martin:1997ns}.
	\item The Standard Model does not explain \textit{Dark Matter} \cite{Pomarol:2012sb}.
	\item The Standard Model requires the neutrinos to be massless. However, the existence of oscillations between neutrinos flavors has been experimentally observed, implying that their mass is in fact non-zero \cite{PhysRevD.22.2227,Fukuda:1998fd,Nustatus}.
\end{itemize}

While the first three issues represent aspects of Nature that are not included in the SM or that remain puzzling, the neutrino mass issue is a serious problem, as the SM directly contradicts the experimental evidence \cite{Fukuda:1998fd}.

Several attempts have been made to find remedies for these issues from a theoretical point of view, and because they come with predictions of new particles, they are subject to experimental examination. The seesaw mechanism aims to provide answers to the question of how neutrinos acquire mass. 

\section{Type-III Seesaw Mechanism}
\subsection{Phenomenology}
\label{sec:Theory/SeesawPhenomenology}

The seesaw mechanism introduces new heavy particles coupling both to leptons and to Higgs doublets, and accounts for both the neutrino masses and their smallness (six or more orders of magnitude smaller than that of the electron) \cite{SeesawI,typeIa,typeIb,typeIe,typeIIa,typeIIb,typeIIc,typeIId,typeIIe,SeesawIII:a,Seesawinverse}.

%These new heavy particles could be weak-singlet fermions (type I \cite{SeesawI,typeIa,typeIb,typeIe}); weak-triplet scalars (type II \cite{typeIIa,typeIIb,typeIIc,typeIId,typeIIe}); or weak-triplet fermions (type III \cite{SeesawIII:a}). They generate a small Majorana mass for the neutrinos, given by, in type I and type III models: $m_{\nu}=Y^T~ \mathcal{M}^{-1}~ Y ~ v^2/2$, where $Y^T$ is the transpose of the Yukawa coupling matrix between the mediators and the SM fermions, $\mathcal{M}$ is the mass matrix of the heavy partner of the neutrinos and $~v$ is the Higgs vacuum expectation value. When mediator mass is large enough (of order of $10^{14}\GeV$), small neutrino masses are generated even for order $O(1)$ Yukawa couplings. If $M$ is smaller (of the order of a few hundreds of \GeV, as in LHC searches), either smaller Yukawa couplings (i.\,e. ``natural'' mixing angles of the order of $10^{-6}$) are required to generate small neutrino masses, or an alternative suppression mechanism is needed (e.\,g. seesaw inverse mechanism \cite{Seesawinverse}). The possible allowed values of the mixing parameters and their products are constrained by precision electroweak data \cite{typeIIIconstraint}.

Within the type-III seesaw model \cite{SeesawIII:a}, the neutrino is considered a Majorana particle whose mass arises via the mediation of massive fermion partners. These massive partners are the fermionic $SU(2)$ triplet of the heavy Dirac charged leptons $\Sigma^\pm$, and the heavy Majorana neutral lepton $\Sigma^0$, coupling both to the leptons and to the Higgs doublets. During proton-proton collisions, the heavy fermion particles may be pair-produced through electroweak interactions in both charged-charged and charged-neutral pairs as can be seen in Fig.~\ref{fig:SeesawProduction}.

\begin{figure}
\begin{center}
	\includegraphics[width=.5\textwidth]{Introduction/SeesawProduction-SpSm}%
	\includegraphics[width=.5\textwidth]{Introduction/SeesawProduction-SpmS0}
	\caption{Examples of Feynman diagrams for heavy fermion production in the type-III seesaw model.
	\label{fig:SeesawProduction}}
\end{center}
\end{figure}

We conduct a search for this signal by examining the final state with at least three electrons or muons. The primary decay channels of interest are $\Sigma^\pm \to \PW^\pm \nu$, $\Sigma^\pm \to \Z \ell^\pm$, $\Sigma^\pm \to \PH \ell^\pm$, $\Sigma^0 \to \PW^\pm \ell^\mp$, $\Sigma^0 \to \Z \nu$, $\Sigma^0 \to \PH \nu$, where $\ell = e, \mu$. Decays of $\Sigma^0 \Sigma^\pm$ and $\Sigma^+ \Sigma^-$ pairs result in 27 different production processes and can naturally lead to multilepton final states if several \PW\ or \Z\ bosons are involved, either directly or via a Higgs boson decay. An example Feynman diagram for one of the most relevant processes with three leptons in the final state, $\Sigma^\pm \Sigma^0 \to \PW^\pm \nu \PW^\pm \ell^\mp$ with leptonic $\PW^\pm$ decays, is shown in Fig.~\ref{fig:SeesawDecay}.
The decay rate of a $\Sigma$ to a given lepton $\ell$ is proportional to $v_{\ell N} = \frac{V_\ell}{\sqrt{|V_e|^2 + |V_\mu|^2 + |V_\tau|^2}}$. In the democratic scenario, the mixing parameters $V_\ell$ are the same for all the leptons so that $v_{\ell N} = \frac{1}{\sqrt{3}}$.

\begin{figure}
\begin{center}
	\includegraphics[width=.5\textwidth]{Introduction/Seesaw}
	\caption{Feynman diagram example of the fermion production and decay in the type-III seesaw model.
	\label{fig:SeesawDecay}}
\end{center}
\end{figure}


\subsection{Signal Model and Generation}
\label{sec:Samples/Signal}

We generate MC events to simulate all 27 production and decay mode combinations (see Sec.~\ref{sec:Introduction}). Generation for the model begins with a FeynRules Model file \cite{SeesawIII_Biggio}. SaloMonte Carlo events are then generated in MadGraph5\_aMC@NLO \cite{Alwall:2011uj}. Bosonic decays are handled through Pythia 8, which is also in charge of hadronization \cite{Sjostrand:2007gs}. At the analysis level, we apply weights to correct for mismodeling of pile-up and \MET resolution.

The production cross sections were calculated with NLO + NLL accuracy using the CTEQ6.6 and MSTW2008nlo90cl parton distribution functions (PDFs) \cite{Fuks:2012qx,Fuks:2013vua}. Flavor-democratic values of the mixing angles are taken ($V_e = V_\mu = V_\tau = 10^{-6}$). This has no direct consequence on the fermion production cross section, but affects the branching ratios. The branching fraction of a heavy fermion to a lepton of flavor $\ell = e, \mu, \tau$ is proportional to $v_{\ell N} = \frac{V_\ell}{\sqrt{|V_e|^2 + |V_\mu|^2 + |V_\tau|^2}}$. The branching ratios from the pair-produced fermions to the bosonic level of the most relevant decay modes are given in Fig.~\ref{fig:SeesawBR}.

\begin{figure}
\begin{center}
	\includegraphics[width=.8\textwidth]{Theory/BR}
	\caption{Branching ratios from the pair-produced fermions to the bosonic level of the most relevant decay modes.
	\label{fig:SeesawBR}}
\end{center}
\end{figure}

\chapter[Experimental Apparatus]{Experimental Apparatus\protect\footnote{This chapter is largely taken from Ref.~\cite{Thomassen2012} (the author's Master's Thesis).}}
\label{chap:Detector}

\section{The Large Hadron Collider}
The particle collisions analyzed in the present thesis were generated by the Large Hadron Collider (LHC) which is located $100\,\unit{m}$ underground in the French--Swiss border area at the outskirts of Geneva \cite{1748-0221-3-08-S08001}. Several pre-accelerators are employed in order to accelerate the protons to different energies and to split them into bunches, before they reach the LHC ring (see Fig.~\ref{fig:LHC}) to form two beams traveling in opposite directions. In this ring of $26.7\,\unit{km}$ circumference, 1232 superconducting dipole magnets are used to produce a magnetic field of up to $8.33\,\unit{T}$ in order to accelerate the protons to their final center of mass energy of $\sqrt{s} = 13\,\TeV$.
%\footnote{The maximum design energy is $14\TeV$. LHC has been operating with increasing energies over the years, and it is planned to reach the design goal in 2014?.}
Additionally, about 7000 magnets are used for trajectory corrections and bunch focusing. Once the final velocity is reached, the protons are directed onto each other at certain points around the accelerator ring, where they collide. The collision products, in general, are not stable, but decay to intermediate and final state particles which are detected by large detector devices such as ATLAS or CMS. The bunch spacing is such that interactions are separated in time by $25\,\unit{ns}$.\footnote{However, several interactions might occur at the same time when two bunches meet. This phenomenon is referred to as ``pile-up'' and must be corrected for at analysis time, mostly by means of geometrical separation of the primary interaction vertex and by subtraction of expected pile-up contributions.}

The design luminosity of LHC is $10^{34}\,\unit{cm^{-2}s^{-1}}$. The instantaneous luminosity is given by
\begin{eqnarray}
	L = \frac{N_p^2 n_b f_\text{rev} \gamma_r}{4\pi \epsilon_n \beta*} F
\end{eqnarray}
where $N_b$ is the number of particles per bunch, $n_b$ is the number of bunches per beam, $f_\text{rev}$ is the revolution frequency, $\gamma_r$ is the relativistic gamma factor, $\epsilon_n$ is the normalized transverse beam emittance, $\beta*$ is the beta function at the collision point, and $F$ is the geometric luminosity reduction factor due to the crossing angle at the interaction point.

\begin{figure}
	\includegraphics[width=.8\textwidth]{Detector/0812015}
	\centering
	\caption{CERN Accelerator Complex \cite{Christiane:1260465}. The diagram shows the different accelerators, detectors, and other facilities at CERN. For proton collisions, not all of the machinery is needed: Protons are initially accelerated to $50\MeV$ in a Linear Accelerator (LINAC~2). Then, they are transported to the Booster ($1.4\GeV$), to the Proton Synchrotron (PS, $25\GeV$) and the Super Proton Synchrotron (SPS, $450\GeV$) from where they are injected into LHC. The PS also takes care of arranging the protons in bunches with the correct spacing for LHC.}
	\label{fig:LHC}
\end{figure}

\section{The CMS Detector}
The Compact Muon Solenoid (CMS) is located at point 5 of the LHC accelerator ring and one of the two large, general purpose detector systems built at LHC. CMS consists of a large superconducting solenoid which contains a silicon-based tracker, an electromagnetic calorimeter made of scintillating lead-tungstate crystals, and a brass-based scintillating hadron calorimeter (see Fig.~\ref{fig:CMS}); the total weight is about 12500 tons \cite{Chatrchyan:2008zzk}. A special feature of CMS is its superconducting solenoid of $6\,\unit{m}$ internal diameter which creates a strong magnetic field ($3.8\,\unit{T}$) that is suitable for high precision measurements of charged particles at very high energies.

\begin{figure}
	\includegraphics[width=\textwidth]{Detector/CMS_Slice}
	\centering
	\caption{A transverse slice through CMS \cite{CMSslice}. The illustration shows the most important detector components as well as examples of different particles as they are detected while traveling through the detector.}
	\label{fig:CMS}
\end{figure}

In order to describe the properties of particles observed in collision events, a coordinate system is defined. The origin is declared where the main interaction point is expected to occur. The $x$ axis points radially towards the center of the LHC, the $y$ axis points in the upward direction, and together with the $z$ axis that points along the beampipe (counterclockwise), a right-handed coordinate system is constructed. In cylindrical coordinates, the $z$ axis is the same, and $\phi$ is the azimuthal angle. Starting from the positive $z$ axis, the polar angle $\theta$ increases towards the center of the LHC. Since the polar angle $\theta$ is not Lorentz-invariant, the pseudorapidity $\eta$ is defined as a Lorentz-invariant alternative coordinate,\footnote{This quantity is the massless limit of the rapidity which is an additive measure of relativistic velocity and defined as $\log \frac{E + p_z}{E - p_z}$.}
\begin{eqnarray}
	\eta = - \log \tan \frac{\theta}{2}.
\end{eqnarray}
When the directional separation between particles needs to be determined,
\begin{eqnarray}
	\Delta R = \sqrt{(\Delta \eta)^2 + (\Delta \phi)^2}
\end{eqnarray}
comes in handy as a measure of two particles' separation in $\eta$ and $\phi$.

The following sections are concerned with the individual detector components used for the measurement of the particle properties that are recorded from collision events, along the lines of Ref.~\cite{Xie:1455454}.

\subsection{Tracking System}
In order to precisely reconstruct the path of charged particles in CMS, a tracking system based on silicon-based p--n junctions was installed. A high reverse-bias voltage is applied across the junction, creating a depletion zone with an electric field. When a charged particle passes this zone, it ionizes the silicon atoms, and the resulting electrons are free to move and create an electrical current which is detected. By setting up several layers containing a large number of such p--n junctions with small dimensions, a highly sensitive tracking device can be created. In total, 15400 tracking sensors are installed in CMS and operated at low temperature in order to minimzie the effects of radiation damage. The CMS tracking system consists of several parts:

\subsubsection*{Pixel Detector}
The Pixel Detector is located within $10\,\unit{cm}$ from the $z$ axis and is used to account for small displacements close to the primary vertex, To keep the occupancy per bunch crossing reasonably low, a pixel size of $100\,\mum \times 150\,\mum$ is used. The spatial resolution is $10\,\mum$ to $20\,\mum$.

\subsubsection*{Strip Detector}
The Strip Detectors are located both in the barrel as well as in the endcap regions of CMS. In either case, several layers of silicon strips are placed behind each other to provide similar functionality as in the case of the pixel detector. The dimensions are much wider than those of the pixels. In each detector region, they are chosen according to the corresponding production characteristics such that the occupancy will not be too high, so that a hit will provide informative value.

\subsection{Electromagnetic and Hadronic Calorimeter}
The calorimetry system is designed to measure the energies of incident particles. Depending on the particle type, the energy is deposited in different parts of the system \cite{EvaHalkiadakis}:

\begin{figure}
	\includegraphics[width=.7\textwidth]{Detector/ECAL_crystal}
	\centering
	\caption{A module of the electromagnetic calorimeter consisting of 500 lead-tungstate crystals.}
	\label{fig:ECAL}
\end{figure}

\subsubsection*{Electromagnetic Calorimeter (ECAL)}
The task of the ECAL is to measure the energy of charged particles (especially electrons) and photons. The lead-tungstate (PbWO$_4$) material of the crystals is very dense, but optically transparent; a module is shown in Fig.~\ref{fig:ECAL}. When electrons or photons travel through, they lose energy in a cascade process due to bremsstrahlung and ionization (electrons) and $e^+ e^-$ pair production (photons). In addition, the crystals are excited so that they produce light from scintillation which is detected by photodetectors and used to infer the incident particle's energy.

In the barrel ($|\eta| \leq 1.479$), there are 61200 crystals with front face dimensions of $22\,\unit{mm} \times 22\,\unit{mm}$, covering 0.0174 in both $\eta$ and $\phi$, and a length of $230\,\unit{mm}$, corresponding to about 25 radiation lengths. In the endcap ($1.479 \leq |\eta| \leq 3.0$), there are 7324 crystals with a surface area of $28.6\,\unit{mm} \times 28.6\,\unit{mm}$ and a length of $220\,\unit{mm}$. An additional preshower detector is installed in front of the endcap component that helps distinguishing photons from neutral pions. This setup covers the $\eta$ range up to the forward region without any gaps.

\subsubsection*{Hadronic Calorimeter (HCAL)}
Like the ECAL, the HCAL is for the most part located inside the solenoid. While the ECAL is a homogeneous, the HCAL is a sampling calorimeter which means that it consists of alternating layers of an active, signal-generating medium, and a passive medium whose only purpose is to absorb energy. The active material is a plastic scintillator which is $3.7\,\unit{mm}$ thick and organized in a tile pattern. The scintillation light emitted in a certain $\eta$--$\phi$ cell is summed up optically, forming a ``tower'', collected by wavelength-shifting fibers, and channeled to hybride photodiodes.

The barrel part ($|\eta| \leq 1.4$) has 2304 towers, each covering 0.087 in $\eta$ and $\phi$. There are 15 absorption layers, mostly made from brass. To increase accuracy, a number of layers is placed at the outside of the magnet coil (Hadron Outer, HO). The endcap parts cover the region $1.3 \leq |\eta| \leq 3.0$ with 19 layers of active scintillating material, covering cell of width $5^\circ$ to $10^\circ$ in $\phi$ and 0.35 to 0.09 in $\eta$. In the very forward region ($3.0 \leq |\eta| \leq 5.0$), a fourth HCAL part (Hadron Forward, HF) consisting of an active quartz fiber medium and steel absorbers is located. The quartz fiber material emits Čerenkov light that is detected by photomultipliers with resolution 0.175 in $\eta$ and $10^\circ$ in $\phi$.

\subsection{Muon System}
The muon is about 200 times as heavy as the electron. Since the bremsstrahlung-induced dissipation in the calorimeter is proportional to $\text{mass}^{-2}$ \cite{bock2013particle}, it is suppressed by a factor of 40000. Therefore, muons can easily traverse the calorimeter system, so that other more specialized detector systems can be employed outside the calorimeter.

In the barrel, 250 drift tube (DT) chambers are used to identify muons. Four shells of stations are located at different distances from the $z$ axis, embedded in the return yoke of the solenoid (see Fig.~\ref{fig:muonSystem}). In the endcap, 468 cathode strip chambers (CSC) are arranged in concentric rings, most of them containing 36 CSCs. Charged particles travelling through the gas inside a CSC cause ionization, followed by a charged avalanche whose distribution is measured on the cathode plane. From this information, it is possible to reconstruct the track geometry. Each of the DT and CSC stations is accompanied by resistive plate chambers that are used for precise timing and velocity determination.

\begin{figure}
	\includegraphics[width=\textwidth]{Detector/muon}
	\centering
	\caption{Sketch of the muon system in CMS in $r$--$z$ view. The drift tubes are displayed in dark-green, the cathode strip chambers in dark-blue, and the resistive plate chambers in dark-red. The light-colored areas are the tracker and calorimeter. The interaction point is located at the origin of the coordinate system.}
	\label{fig:muonSystem}
\end{figure}

\subsection{Trigger and Data Storage}
LHC performs about about $10^7$--$10^8$ proton--proton collisions per second. Since not all events can be stored (about $300\,\text{Hz}$), a rejection rate of about $10^5$ is required. First-level decisions are reached by the Level-1 (L1) trigger system which performs quick assessments of events within about 1\mus while the event data, about $0.5\,\text{MB}$ each, is held in buffers. Potentially interesting events are then forwarded to a dedicated computing farm where high-level triggers (HLT) run more precise reconstruction algorithms in order to decide which events should be kept.

Finally, accepted events are transmitted to the storage manager system which arranges the subsequent transfer to the permanent Tier-0 storage systems located at the CERN main site and at the Wigner Research Centre for Physics in Budapest (Hungary). From there, data is distributed to interested Tier-1 and Tier-2 sites across the globe for analysis purposes. Petabyte-range storage systems are employed world-wide to manage the large amounts of data that are used on a daily basis.

\chapter{Datasets and Triggers}
\label{chap:Samples}

\section{Triggers}
The data for this search are collected using several dilepton triggers. The double electron trigger requires two electrons with \pt thresholds of 17\GeV on the leading electron and 12\GeV on the sub-leading electron. The double muon trigger requires two muons with \pt thresholds of 17 and 8\GeV on the leading and sub-leading muons, respectively. We use two muon/electron cross triggers, one of which requires a 17\GeV muon and a 12\GeV electron, while the other requires a 17\GeV electron and a 8\GeV muon.

We use the data sets listed in Table~\ref{tab:DataSamples}, masked using the ``Golden JSON file'' \path{/afs/cern.ch/cms/CAF/CMSCOMM/COMM_DQM/certification/Collisions15/13TeV/Reprocessing/Cert_13TeV_16Dec2015ReReco_Collisions15_25ns_JSON_v2.txt}.

\begin{table}
\centering
\caption{Data samples.} \label{tab:DataSamples}
\begin{tabular}{c c c}
\hline\hline
Primary Dataset & Reconstruction labels & L [fb$^{-1}$]\\
\hline
DoubleEG & Run2015D-05Oct2015-v1 & 0.59\\
DoubleEG & Run2015D-PromptReco-v4 & 1.66\\
\hline
DoubleMuon & Run2015D-05Oct2015-v1 & 0.59\\
DoubleMuon & Run2015D-PromptReco-v4 & 1.66\\
\hline
MuonEG & Run2015D-05Oct2015-v1 & 0.59\\
MuonEG & Run2015D-PromptReco-v4 & 1.66\\
\end{tabular}
\end{table}


\section{Background MC samples}
For background determination, we use the Monte Carlo samples listed in Table~\ref{tab:MCSamples}.

\begin{sidewaystable}
\centering
\caption{Background MC samples.} \label{tab:MCSamples}
\begin{tabular}{p{14cm} l c c c l}
\hline\hline
Sample & xsec [pb] & L [pb$^{-1}$] & No. events read\\
\hline
/WZTo3LNu\_TuneCUETP8M1\_13TeV-powheg-pythia8 \newline /RunIISpring15DR74-Asympt25ns\_MCRUN2\_74\_V9-v1/MINIAODSIM	& 4.42965	& 447169	& 1.9808e+06\\
\hline
/WZJets\_TuneCUETP8M1\_13TeV-amcatnloFXFX-pythia8/ \newline RunIISpring15MiniAODv2-74X\_mcRun2\_asymptotic\_v2-v1/MINIAODSIM	& 5.263	& 2.37929e+06	& 1.252220e+07\\
\hline
/ZZTo4L\_13TeV-amcatnloFXFX-pythia8 \newline /RunIISpring15DR74-Asympt25nsRaw\_MCRUN2\_74\_V9-v1/MINIAODSIM	& 1.212	& 8.71378e+06	& 1.05611e+07\\
\hline
/TTTo2L2Nu\_13TeV-powheg \newline /RunIISpring15DR74-Asympt25ns\_MCRUN2\_74\_V9-v1/MINIAODSIM	& 87.31	& 69711.1	& 4.997e+06\\ 
\hline
/TTWJetsToLNu\_TuneCUETP8M1\_13TeV-amcatnloFXFX-madspin-pythia8 \newline /RunIISpring15DR74-Asympt25ns\_MCRUN2\_74\_V9-v1/MINIAODSIM	& 0.2043	& 1.23792e+06	& 252908\\
\hline
/TTZToLLNuNu\_M-10\_TuneCUETP8M1\_13TeV-amcatnlo-pythia8 \newline /RunIISpring15DR74-Asympt25ns\_MCRUN2\_74\_V9-v1/MINIAODSIM	& 0.2529	& 1.57374e+06	& 398000\\
\hline
/WWZ\_TuneCUETP8M1\_13TeV-amcatnlo-pythia8/ \newline RunIISpring15MiniAODv2-74X\_mcRun2\_asymptotic\_v2-v1/MINIAODSIM	& 0.1651	& 1.51423e+06	& 250000\\
\hline
/WZZ\_TuneCUETP8M1\_13TeV-amcatnlo-pythia8/ \newline RunIISpring15MiniAODv2-74X\_mcRun2\_asymptotic\_v2-v1/MINIAODSIM	& 0.05565	& 4.49236e+06	& 250000\\
\hline
/ZZZ\_TuneCUETP8M1\_13TeV-amcatnlo-pythia8/ \newline RunIISpring15MiniAODv2-74X\_mcRun2\_asymptotic\_v2-v1/MINIAODSIM	& 0.01398	& 1.78827e+07	& 250000\\
\hline
/GluGluHToZZTo4L\_M125\_13TeV\_powheg\_JHUgen\_pythia8 \newline /RunIISpring15MiniAODv2-74X\_mcRun2\_asymptotic\_v2-v1/MINIAODSIM	& 0.01212	& 4.10891e+07	& 498000\\
\hline
/VBF\_HToZZTo4L\_M125\_13TeV\_powheg\_JHUgen\_pythia8 \newline /RunIISpring15MiniAODv2-74X\_mcRun2\_asymptotic\_v2-v1/MINIAODSIM	& 0.001034	& 4.73362e+08	& 489456\\
\hline
/DYJetsToLL\_M-10to50\_TuneCUETP8M1\_13TeV-amcatnloFXFX-pythia8 \newline /RunIISpring15MiniAODv2-74X\_mcRun2\_asymptotic\_v2-v1/MINIAODSIM	& 18610	& 1613.2	& 3.002156+e07\\
\hline
/DYJetsToLL\_M-50\_TuneCUETP8M1\_13TeV-amcatnloFXFX-pythia8 \newline /RunIISpring15MiniAODv2-74X\_mcRun2\_asymptotic\_v2-v1/MINIAODSIM	& 6025.2	& 4771.29	& 2.874797+e07\\

\end{tabular}
\end{sidewaystable}

Some MC generators provide events with negative weights to allow for a more precise prediction of higher-order contributions. Where provided, we take these negative weights into account.

For each MC sample, the pile-up distribution is compared to the distribution in data. Weights are applied on a per-event basis so that the distribution matches for each sample.

We also correct the pile-up dependence of our data-driven background estimate using the linear fit $0.773 + 0.0218 \cdot n_\textrm{vertex}$. As a result of these weights, the background shape of the $n_\textrm{vertex}$ distribution agrees with the data. Fig.~\ref{fig:pileupWeights} shows this distribution before and after pile-up weights for events with at least three electrons or muons, without any other cuts.

\begin{figure}[h]
\begin{center}
	\begin{subfigure}[b]{.7\textwidth}
		\includegraphics[width=\textwidth]{Samples/NVERTICES_Tau0-noPileupWeights}
		\caption{before pile-up weights}
	\end{subfigure}
	\begin{subfigure}[b]{.7\textwidth}
		\includegraphics[width=\textwidth]{Samples/NVERTICES_Tau0}
		\caption{after pile-up weights}
	\end{subfigure}
	\caption{$n_\textrm{vertex}$ distribution for events with at least three light leptons.
	\label{fig:pileupWeights}}
\end{center}
\end{figure}

\chapter{Selection}
\label{chap:Selection}

\section{Object Selection}
\label{sec:Selection/Object}

\subsection{Electrons and Muons}
Events that pass the trigger are required to satisfy additional selection criteria. Electrons with $\pt \geq 7\GeV$ and $|\eta| \leq 2.5$ as well as muons with $\pt \geq 5\GeV$ and $|\eta| \leq 2.4$ are reconstructed using the particle-flow (PF) algorithm which utilizes measured quantities from the tracker, calorimeter, and muon system \cite{CMS-PAS-PFT-09-001}. The matching candidate tracks must satisfy quality requirements and spatially match with the energy deposits in the ECAL and the tracks in the muon detectors, as appropriate.

Sources of background leptons include genuine leptons occurring inside or near jets, hadrons that punch through into the muon system and are misidentified as muons, hadronic showers with large electromagnetic fractions, or photon conversions. Since the leptons from the seesaw signal are generally not related to hadronic activity, electrons and muons are expected to be spatially isolated from any jets occurring in the event. Furthermore, as jets occasionally produce leptons in their vincinity, an isolation requirement strongly reduces the background from such misidentified leptons.

The isolation requirement imposes a selection based on the size of the lepton transverse momentum in comparison to the transverse momenta of other particles in its immediate neighborhood. A common isolation discriminator is the the ``relative isolation'',
\begin{equation}
	I_\text{rel}(\Delta R^\textrm{max}) = \frac{\sum_\text{other}^{\Delta R < \Delta R^\textrm{max}} \pt^\text{other}}{\pt^\ell},
\end{equation}
defined as the transverse momentum sum of charged hadrons, neutral hadrons, and photons within a $\Delta R^\textrm{max}$ cone around the lepton candidate, divided by the lepton's own \pt.

However, as the masses of the heavy fermions in the seesaw model are high, their decay products are expected to be boosted and thus harder to distinguish geometrically. The relative isolation variable therefore does not provide good enough separation. We therefore use the multi-isolation discriminator which has better discrimination power in the case of boosted topologies \cite{CMS-PAS-SUS-15-008}. This is achieved using the following three input variables:

\begin{enumerate}
	\item A modified version of the relative isolation $I_\textrm{rel}$, called mini-isolation (\miniIso) \cite{Rehermann:2010vq}. Just like $I_\textrm{rel}$, it is the ratio of the scalar sum of the transverse momenta of charged hadrons, neutral hadrons, and photons within a cone around the lepton candidate; however, the cone radius depends on the transverse momentum of the lepton candidate itself as
		\begin{equation}
			\Delta R^\textrm{max}(\pt^\ell) = \dfrac{10\GeV}{\min\left[\max\left(\pt^\ell, 50\GeV\right), 200\GeV\right]}.
		\end{equation}
		The cone size thus varies between 0.2 and 0.05 as a function of the lepton \pt and is smaller for higher values of the lepton \pt. Hence, on the one hand, we reduce the chance of overlap with other objects as the lepton becomes stiffer. On the other hand, as misidentified leptons usually have rather low \pt, the larger cone size for low-\pt leptons improves the background rejection efficiency.
		
	\item We use the ratio of the lepton \pt and the \pt of a jet in which the lepton is contained:
		\begin{equation}
			\ptRatio = \dfrac{\pt^\ell}{\pt^\text{jet}}.
		\end{equation}
		Note that every lepton is, by definition, contained in a jet (which may contain nothing else besides the lepton). The \ptRatio variable thus tells us the relative share of jet momentum that is associated with the lepton. The larger this fraction, the lesser is the chance that the lepton stems from a hadronic decay or is misidentified. This variable is similiar to $I_\textrm{rel}$, except that the cone is replaced by the jet.
		
	\item To avoid rejecting leptons that fail the \ptRatio requirement because they overlap accidentally with another jet in the event, we consider the lepton candidate \pt along the axis of the residual momentum of the closest jet after subtracing the lepton \pt:
		\begin{equation}
			\ptRel=\frac{(\vec{p}(\text{jet})-\vec{p}(\ell))\cdot \vec{p}(\ell) }{|\vec{p}(\text{jet})-\vec{p}(\ell)|}.
		\end{equation}
		If the \ptRatio condition is not met while the lepton still has substantial momentum along the residual momentum axis, the lepton is allowed to pass.
\end{enumerate}
A lepton is considered to be isolated if the following condition is met:
\begin{equation}
	\miniIso < I_1 \wedge ( \ptRatio > I_2 \vee \ptRel > I_3 )
\end{equation}

The values of $I_\text{i}$, $i = 1,2,3,$ depend on the lepton flavor. For electrons, the medium working point is employed, while for muons, we use the tight working point. As the chance of misidentification is higher for electrons, tighter isolation values are used in this case (see Table~\ref{tab:isoWPs}). Further details on the input variables, these and other working points, and their efficiencies can be found in \cite{CMS-PAS-SUS-15-008}.

\begin{table}
\centering
\caption{Multi-isolation working points used in the analysis.} \label{tab:isoWPs}
\begin{tabular}{l ccc}
\hline\hline
Isolation value & Muons & Electrons  \\
\hline
$I_1$ & 0.20 & 0.16 \\
$I_2$ & 0.69 & 0.76 \\
$I_3$ [GeV] & 6.0 & 7.2 \\
\hline
\end{tabular}
\end{table}

The signal leptons originate from the interaction point. After the isolation selection, the most significant background sources are residual non-prompt leptons from heavy quark decays, where the lepton tends to be more isolated because of the high \pt with respect to the jet axis. This background is reduced by requiring that the leptons satisfy $d_\textrm{z} \leq 0.1\,\cm$ where $d_\textrm{z}$ is the longitudinal impact parameter with respect to the primary interaction vertex, and that the impact parameter $d_\textrm{xy}$ between the track and the event vertex in the plane transverse to the beam axis be small: $d_\textrm{xy} \leq 0.05\,\cm$. The isolation and impact parameter criteria retain signal but significantly reject misidentified leptons.

\subsection{Missing Transverse Momentum (\MET)}
The missing transverse momentum is calculated as the negative vectorial sum of the transverse momenta of all the PF candidates. The missing transverse energy \MET is defined as the magnitude of this vector. Jet energy corrections are applied to all jets and also propagated to the calculation of \MET \cite{CMS-PAS-JME-12-002}. We apply additional smearing to simulation samples to model the \MET resolutions we find in data as a function of jet activity and the number of interaction vertices in an event.


\section{Event Selection}
\label{sec:Selection}

For the three leading leptons, we apply offline thresholds of 20, 15, 10\GeV. We find that with these thresholds, the trigger efficiency of trilepton events is close to 100\%.

Events with an opposite-sign lepton pair with mass below 12 GeV are vetoed to reduce background from low-mass resonances.
%Furthermore, we reject trilepton events with an OSSF pair below the \Z boson mass window when the trilepton mass is within the \Z mass window. This cuts away background from asymmetric photon conversions in $\Z \to \ell\ell^* \to \ell\ell\gamma$, where the photon converts into two additional leptons, one of which is lost.

\chapter{Search Strategy}
\label{sec:Strategy}

Candidate events in this search must have a total of at least three leptons, each of which can be either an electron or a muon. We classify multilepton events into search channels on the basis of the number of leptons, lepton flavor, lepton relative charges, charge and flavor combinations, and other kinematic quantities described below.

We classify each event in terms of the maximum number of opposite-sign same-flavor (OSSF) dilepton pairs that can be made by using each lepton only once. For example, both $\mu^+\mu^-\mu^-$ and $\mu^+\mu^-e^-$ are OSSF1, $\mu^+\mu^+e^-$ is OSSF0, and $\mu^+\mu^-e^+e^-$ is OSSF2. We denote a lepton pair of different flavors as $\ell\ell^\prime$.

We classify events as containing a leptonically-decaying \Z if at least one OSSF pair has $m_{\ell^+\ell^-}$ in the \Z mass window $91 \pm 10\,\GeV$. For $m_{\ell^+\ell^-}$ outside the \Z boson mass window, events are separated into bins below and above the \Z mass window. In cases of ambiguity (such as $\mu^+\mu^-\mu^-$), the pair below the \Z mass window takes precedence (thus shifting events from high mass to low mass, for a more separative background categorization). We refer to these three mass ranges as ``on-\Z'', ``below-\Z'', and ``above-\Z''.

The most important multilepton background processes are \WZ, \Z or \ttbar events in which there is a misidentified lepton, and \ZZ production. In addition, there are various rare background processes like WWZ or $\ttbar\PW$. However, the level of SM background varies considerably across channels; for example, channels containing OSSF pairs suffer from larger backgrounds than do channels with OSSF0. Hence, all these charge combinations are considered as different channels.

%Backgrounds can be tamed by binning in appropriate quantities. Since the signal leptons have relatively high \pt and since in some decay modes there is missing transverse energy (\MET) from accompanying neutrinos, we find that binning in $L_\textrm{T} + \MET$, where $L_\textrm{T}$ is the scalar sum of the lepton \pt's, separates the signal from the background. We find that lepton \pt binning alone gives about 20\,\% worse signal-to-background ratios in the most sensitive signal regions.
%Additional separation is achieved through the on- and off-\Z binning described above.

\chapter{Backgrounds}
\label{chap:Backgrounds}

To judge the importance of each background, we consider the breakdown in the ten signal regions that are most sensitive to the signal. These signal regions, such as the region with three leptons, an OSSF pair on-\Z, and $L_\textrm{T} + \MET > 550\,\GeV$, lie in the peripheral areas of the multilepton phase space and are thus limited by statistics. The most notable backgrounds are:
\begin{enumerate}
	\item $\WZ \to \ell\ell\ell$. This process is responsible for about 51\,\% of the total background in the top ten signal regions (i.\,e. the 10 most sensitive bins of Fig.~\ref{fig:Results}).
	\item Fully leptonic \ttbar decays with a misidentified lepton from a b-jet, 21\,\% of the total background.
	\item $\Z \to \ell\ell$ plus a misidentified lepton from a jet or a photon. This process makes up about 17\,\% of the total background.
	\item $\ZZ \to 4\ell$, 3\,\% of the total background.
\end{enumerate}

The prompt diboson backgrounds (\WZ and \ZZ) are obtained from simulation, but normalized and validated in data control regions. For processes that contain misidentified leptons (\Z or \ttbar accompanied by a third lepton), misidentification rates are measured in appropriate control regions. The \Z + fake estimate is fully data-driven using a method that also covers similar, albeit smaller backgrounds like \PW\PW\ + fake. In our figures, this background is labeled ``Misidentified''.

In the case of \ttbar, the process-specific kinematics are harder to capture using a fully data-driven method; we thus extract the kinematics from MC, while the misidentification rate remains data-driven. The remaining 9\,\% of the background are due to rare processes like $\ttbar\Z$, $\ttbar\PW$, and $\PH \to 4\ell$ which we obtain directly from MC simulation. In our figures, these backgrounds are denoted ``Rare MC'' and ``Higgs'', respectively.

Whenever MC simulation is used, we rely on the Powheg or MadGraph5\_aMC@NLO generators. An overview of the control regions involved in our background studies is given in Table~\ref{tab:CR}.

\begin{table}
\centering
\caption{Background control regions (left) are defined by the criteria listed at the top. \ST is the scalar sum of the lepton transverse momenta, the transverse momenta of jets, and \MET.} \label{tab:CR}
\begin{tabular}{c | c l c c c }
\hline\hline
 & $n_\textrm{leptons}$ & OS pair & $n_\textrm{b-tags}$ & \ST [GeV] & \MET [GeV] \\
\hline
\ttbar & 2 & 1 opposite flavor & $\geq 1$ & $> 300$ \\
\Z + fake & 3 & 1 same flavor, on-\Z & & & $< 50$ \\
\WZ & 3 & 1 same flavor, on-\Z & & & 50--150 \\
\ZZ & $\geq 4$ & 2 same flavor, at least one on-\Z & & & $< 50$ \\
\hline
\end{tabular}
\end{table}

\section{\texorpdfstring{\WZ}{WZ} Background}
\label{sec:bkg_WZ}

This is the primary background in our search (about 51\,\%). To estimate this process, we define the \WZ control region by 3 leptons, an on-\Z OSSF pair, and $50\GeV < \MET < 100\GeV$. We use \WZ MC with fully leptonic decays and normalize the total number of events in the control region, after subtracting other backgrounds. The normalization factor is $0.95 \pm 0.07\stat$.

We validate the $n_\textrm{jets}$ distribution in the control region (Fig.~\ref{fig:WZ/NGOODJETS}) and find that $n_\textrm{jets}$ weights do not need to be applied. The \WZ-specific shape of the transverse mass distribution (Fig.~\ref{fig:WZ/MET50to100_MT}) is checked as well, where the transverse mass $M_\textrm{T}$ is defined as
$$M_\textrm{T} = \sqrt{2 \MET p_\textrm{T}^\ell \left( 1 - \cos\measuredangle(\vec E_\textrm{T}^\text{miss}, \vec p_\textrm{T}^{\,\ell}) \right)},$$ and $\ell$ refers to the lepton that is not part of the OSSF pair. In case of ambiguity, the OSSF pair is defined as the one whose invariant mass is closer to the \Z boson mass.

\begin{figure}
\begin{center}
	\includegraphics[width=.7\textwidth]{Background/bkg_WZ/WZ_MET50to100_NGOODJETS}
	\caption{$n_\textrm{jets}$ distribution in the \WZ-dominated control region (last bin includes overflow). Uncertainty bands include both statistical and systematic uncertainties, with the exception of the \WZ normalization uncertainty.
	\label{fig:WZ/NGOODJETS}}
\end{center}
\end{figure}

\begin{figure}
\begin{center}
	\begin{subfigure}[b]{.7\textwidth}
		\includegraphics[width=\textwidth]{Background/bkg_WZ/WZ_MET50to100_MT}
		\caption{$50\GeV < \MET < 100\GeV$} \label{fig:WZ/MET50to100_MT}
	\end{subfigure}
	\begin{subfigure}[b]{.7\textwidth}
		\includegraphics[width=\textwidth]{Background/bkg_WZ/WZ_MET100to150_MT}
		\caption{$100\GeV < \MET < 150\GeV$} \label{fig:WZ/MET100to150_MT}
	\end{subfigure}
	\caption{$M_\textrm{T}$ distributions in the \WZ-dominated control and validation regions (last bin includes overflow). Uncertainty bands include both statistical and systematic uncertainties, with the exception of the \WZ normalization uncertainty.
	\label{fig:WZ}}
\end{center}
\end{figure}

In the adjoining $100\GeV < \MET < 150\GeV$ validation region, we find that the normalization is off. Based on the amount of data investigated, we cannot tell whether this is a statistical fluctuation or a mismodeling effect. Detailed investigations of the extra events did not uncover any unexpected kinematic patterns. We thus assign a systematic uncertainty of 50\,\% based on the variation of the normalization factor between the normalization region and the validation region (see Fig.~\ref{fig:WZ/MET100to150_MT}).

\section{\texorpdfstring{\ttbar}{TTbar} Background}
\label{sec:bkg_tt}

The \ttbar process contributes about 21\,\% to the total background in our search. We rely on MC to model the process-specific kinematic properties. To that effect, we first verify that the simulation works well in dilepton events, and then correct the MC misidentification rate to match the one in data.

\subsection{Dilepton Studies}
The \ttbar control region is defined by exactly 2 opposite-sign opposite-flavor leptons ($e^\pm \mu^\mp$), at least 1 b-tagged jet above 30\,\GeV, and $\ST > 300\,\GeV$, where \ST is the scalar sum of the lepton transverse momenta, the transverse momenta of jets with $\pt \geq 30\,\GeV$ and $|\eta| \leq 2.4$, and \MET. We use this region to normalize the background prediction; the normalization factor is $0.80 \pm 0.01\stat$.

We also use this control region to derive weights in bins of $n_\textrm{jets}$ (see Fig.~\ref{fig:tt/NGOODJETS}). The corrections typically range between 1\,\% and 10\,\%. Note that, as we do not bin our signal regions in jet-related quantities, the impact of these weights on the results is neglibile. Still, as the $n_\textrm{jets}$ distribution was seen to be in disagreement, weights were applied to improve the general accuracy of the simulation.

\begin{figure}
\begin{center}
	\begin{subfigure}[b]{.7\textwidth}
		\includegraphics[width=\textwidth]{Background/bkg_tt/ttbar_NGOODJETS_STgt300_beforeWeights}
		\caption{$n_\textrm{jets}$ distribution before weights}
	\end{subfigure}
	\begin{subfigure}[b]{.7\textwidth}
		\includegraphics[width=\textwidth]{Background/bkg_tt/ttbar_NGOODJETS_STgt300_afterWeights}
		\caption{$n_\textrm{jets}$ distribution after weights}
	\end{subfigure}
	\caption{$n_\textrm{jets}$ distributions in \ttbar-dominated control region (last bin includes overflow). Uncertainties are statistical only. \fixme{style}
	\label{fig:tt/NGOODJETS}}
\end{center}
\end{figure}

The \MET and \ST distributions after normalization and weights are shown in Fig.~\ref{fig:tt/METST}.

\begin{figure}
\begin{center}
	\begin{subfigure}[b]{.7\textwidth}
		\includegraphics[width=\textwidth]{Background/bkg_tt/ttbar_MET_STgt300_afterWeights}
		\caption{\MET distribution}
	\end{subfigure}
	\begin{subfigure}[b]{.7\textwidth}
		\includegraphics[width=\textwidth]{Background/bkg_tt/ttbar_ST_STgt300_afterWeights}
		\caption{\ST distribution}
	\end{subfigure}
	\caption{Kinematic distributions in \ttbar-dominated control region (last bin includes overflow). Uncertainties are statistical only.
	\label{fig:tt/METST}}
\end{center}
\end{figure}

\subsection{Trilepton Studies}
\label{sec:bkg_tt/trilepton}

Having established the validity of kinematic aspects of the \ttbar simulation using the dilepton (prompt) sample, we verify that also the rate at which the \ttbar MC produces trilepton events is in agreement with that rate in data. To this end, we measure the lepton misidentification rate in a sample dominated by semi-leptonic \ttbar decays. This sample is selected by requiring one tight muon with $\pt > 30\,\GeV$, 2 jets (from the other top quark), one additional b-tagged jet, and a non-prompt lepton which is most likely misidentified.

We find the \ttbar misidentification rate in data to be $1.5 \pm 0.5\stat$ times the one found in the \ttbar simulation. This indicates that the number of events with misidentified leptons is underpredicted in the \ttbar simulation. We thus use this ratio to correct the number of predicted trilepton events from \ttbar and apply a systematic uncertainty of 50\,\%.

\subsection{Dilepton + Track Studies}
\label{sec:bkg_tt/dilepton+track}

The prediction for the \Z + jets background (Sec.~\ref{sec:bkg_fakeLight}) uses isolated tracks as a handle to estimate the number of misidentified leptons from jets. However, some of these tracks may come from \ttbar + jets production, so that the two methods overlap. To avoid overpredicting the background in the \Z + jets method, the number of tracks as predicted by the \ttbar simulation is subtracted before calculating the track-based estimate. The reliability of this procedure depends on the accuracy of the track modeling in simulation.

We therefore verify that the number of tracks from \ttbar is predicted correctly, by comparing both the distributions of non-isolated tracks and of isolated tracks in the dilepton control region. While the former, shown in Fig.~\ref{fig:tt/trackNonIso}, agrees well with the data and gives us confidence in the general quality of the simulation, the latter shows disagreement in the bins with at least one isolated tracks (Fig.~\ref{fig:tt/trackIso}). We therefore scale the number of tracks from the \ttbar simulation by a factor of 1.5.

Considering both plots in Fig.~\ref{fig:tt/track} together suggests that the \ttbar simulation does not model the track isolation distribution distribution correctly. As the mismodeling of the \ttbar misidentification rate (Sec.~\ref{sec:bkg_tt/trilepton}) is quantitatively and qualitatively similar, it is likely that both discrepancies have a common cause in the simulation. We therefore apply the same systematic uncertainty (50\,\%) to the track prediction.

\begin{figure}
\begin{center}
	\begin{subfigure}[b]{.7\textwidth}
		\includegraphics[width=\textwidth]{Background/bkg_tt/ttbar_NPROMPTNONISOINCLUSIVETRACKS7_STgt300}
		\caption{Number of non-isolated tracks \fixme{label}} \label{fig:tt/trackNonIso}
	\end{subfigure}
	\begin{subfigure}[b]{.7\textwidth}
		\includegraphics[width=\textwidth]{Background/bkg_tt/ttbar_NGOODTRACKS_STgt300}
		\caption{Number of isolated tracks \fixme{label, style}} \label{fig:tt/trackIso}
	\end{subfigure}
	\caption{Track distributions in \ttbar-dominated control region (last bin includes overflow). Uncertainties are statistical only.
	\label{fig:tt/track}}
\end{center}
\end{figure}

Further details on the \Z + jets background estimation can be found in Sec.~\ref{sec:bkg_fakeLight}.

\section{\texorpdfstring{\Z}{Z} + jets Background}
\label{sec:bkg_fakeLight}

The \Z + jets process contributes about 17\,\% of the total background. Since the misidentified leptons are not modeled with sufficient precision by simulation, we employ a data-driven method that uses correlated objects in order to predict the \Z + jets background.

In order to determine the background with misidentified electrons and muons from jets, we select events with the same selection as the signal except requiring one less lepton and requiring an isolated track instead. The isolation criteria that we require these tracks to satisfy are identical to our muon isolation criteria. We verify that the kinematic properties of the isolated tracks resemble those of the misidentified leptons.

The number of 3$\ell$ events in data per 2$\ell$ + track event in this sample then gives the lepton misidentification rate, $\frac{N(3\ell)}{N(2\ell + \textrm{track)}}$. Since we subtract contributions from other backgrounds in the numerator and the denominator, it describes the number of misidentified leptons as a fraction of the number of 2$\ell$ + track events from all processes that are not modeled otherwise.

\label{sec:bkg_fakeLight/jets}
The misidentification rate is measured using events with 3 electrons or muons including an OSSF pair on-\Z and $\MET < 50\,\GeV$. This is the prominent \Z peak region with an additional lepton. In this region, we measure the ratio of the number of 3$\ell$ and 2$\ell$+track events (misidentification rate).

We find the electron and muon misidentification rates to be $(1.59 \pm 0.15\stat)\,\%$ and $(1.49 \pm 0.13\stat)\,\%$, respectively. We apply a systematic uncertainty of 14\,\% to cover the variation of observed misidentification rates as a function of the flavor of the remaining prompt lepton pair in the event.

To apply the misidentification rate in a signal region, we multiply it by the total number of events found in the corresponding 2$\ell$ + track region in data. However, since we use MC to obtain the misidentified contribution for the \ttbar background, we need to correct for double-counting. We thus subtract the contribution from 2$\ell$ + track events as predicted by the \ttbar MC from the data.
%In rare cases due to statistical fluctuations, the subtraction might yield a (small) negative number. If that happens, we replace it by zero, to make sure that the background prediction behaves physically reasonably.


\subsection{Method}
\label{sec:bkg_fakeLight}

To determine the background with fake electrons and muons, we rely on looser objects measured in data that are emitted in a similar way in the decay chain and are therefore expected to be correlated with the fake leptons, and use them as lepton proxies.\footnote{These looser objects are not necessarily leptons as well. For example, a photon that converts into two leptons, one of which has very low \pt, may have kinematics which are very similar to the ones of the other conversion lepton that carries most of the \pt. (Of course, the selection of such objects may be tricky.)} We verify that the kinematic properties of these proxies resemble those of the fake leptons. We then generate a fake sample based on the 2$\ell$+[proxy object] data, treating the proxy objects as leptons (``seed sample''). Further down in the analysis chain, these fake leptons appear just as regular leptons (\eg when computing invariant masses). Proxy objects that can take multiple roles are considered the appropriate number of times (see Sec.~\ref{sec:bkg_fakeLight/jets}).

The number of 3$\ell$ events in data per 2$\ell$+[proxy object] event in this fake sample is then evaluated (``fake rate''). With the help of the fake rate, we predict the background in our signal regions, by applying it to the corresponding seed sample which requires one less lepton and a proxy object instead. Because the proxy objects appear as leptons, this is simply done by selecting the signal region from the fake sample.

To compute the fake rate $\frac{N(3\ell)}{N(2\ell + \textrm{[proxy object])}}$, we subtract contributions from other backgrounds in the numerator and the denominator. This step interacts with the MC background normalizations and thus requires an iterative process to converge. The fake rate then describes the number of fake leptons as a fraction of the number of 2$\ell$+[proxy object] events from all processes that have not been modeled otherwise.

When we apply the fake rate in a signal region, we multiply it by the total number of 2$\ell$+[proxy object] events found in the corresponding seed region in data, without any subtractions from the data sample. However, we use MC to obtain the fake contribution for certain backgrounds.\footnote{This is especially important for \ttbar when a b-tag is not present, since the fake rate is higher in \ttbar events, but there is no obvious way to discern these events from non-\ttbar events in the seed sample.} In these cases, double-counting needs to be mitigated. Therefore, we take the 2$\ell$+[proxy object] component of the background MC sample, apply the same fake rate as for data, and subtract the resulting prediction from the regular data-driven prediction (see \eg Sec.~\ref{sec:bkg_tt} for \ttbar). This is equivalent to keeping the seed sample clean of proxies originating from processes that are modeled otherwise. In rare cases due to statistical fluctuations, the subtraction might yield a (small) negative number. If that happens, we replace it by zero, to make sure that the background prediction behaves physically reasonably.
%\footnote{Another option would be to subtract the MC-fake-seed-driven background from the regular \ttbar MC prediction (again with a lower bound at 0). However, 8\,\TeV cross-checks have shown that this leads to less accurate results. Subtracting from the data-based prediction instead also seems more natural, as this amounts to a pruning of \ttbar-type events from the seed sample that we don't want to predict using the data-driven method.}

%We also study to what extent the fake rate depends on other properties of the event (for example the jet composition and spectra), and parameterize the fake rate as necessary. The freedom that we find in determining these parameterizations and kinematic weights is used to assess the systematic uncertainty of the background estimate.

\subsection{Fake leptons from asymmetric internal photon conversions (AIC)}
\label{sec:bkg_fakeLight/photons}
We look at the number of events that have 3 light leptons (no $\tau_\textrm{had}$) including an OSSF pair below \Z (\ie $m_{\ell\ell} < 81\,\GeV$), no b-tags, $\HT < 200\,\GeV$, and $\MET < 50\,\GeV$. This is essentially the \Z peak region, except that the dilepton invariant mass is not large enough to fall on the \Z peak, and a third lepton is present. This region primarily contains events from $\Z \to \ell\ell$ where one of the final state leptons radiates an off-shell photon which decays, or equivalently internally converts, asymmetrically to two additional leptons, one of which carries very low \pt and is not reconstructed as an independent object in the detector. The process of emission of an off-shell photon through asymmetric internal conversion then yields a single reconstructed lepton in the detector. Since the \pt of the lost lepton is low, the leading three leptons nearly reconstruct the invariant mass of the Z peak. The internal conversion process has an infrared singularity, so the distribution of off-shell photon masses is peaked at very low values. The resulting kinematic distribution in this region of phase space is then very similar to the emission of a real on-shell photon. 

We may therefore form a seed sample with photons as proxies for fake leptons coming from asymmetric internal conversion. All combinations are taken into account, \ie dilepton events with a photon enter the fake sample as four event types (two possible flavors, two possible charges). The photons are required to be within $dR = 0.30..0.60$ from another light lepton, as this is the characteristic distance for radiated photons of the type considered. \fixme{Show plot}

Looking in the seed sample, we find that the $2\ell+\gamma$ mass indeed reproduces the \Z peak, as shown in Fig.~\ref{fig:fakeLight_AIC_MLEPTONS}. %Note: Here, we use the \Z window range 75..100\,\GeV as the method does not model the $m_{\ell\ell\ell}$ shape correctly around 80\,\GeV (see Fig.~\ref{fig:fakeLight_AIC_MLEPTONS_fine}). \fixme{Study how that can be improved}

After applying additional corrections as described in the following paragraphs, we find that the photon fake rates are
\begin{itemize}
	\item muons: 1.60\,\% ($ee$ environment), 1.05\,\% ($\mu\mu$ environment),
	\item electrons: 3.5\,\% ($ee$ environment), 4.5\,\% ($\mu\mu$ environment).
\end{itemize}

\begin{figure}
\begin{center}
	\includegraphics[width=.5\textwidth]{Background/bkg_fakeLight/AIC_MLIGHTLEPTONS_muFake}%
	\includegraphics[width=.5\textwidth]{Background/bkg_fakeLight/AIC_MLIGHTLEPTONS_elFake}
	\caption{$m_{3\ell}$ distribution in AIC-dominated control region. \enskip left)~fake muon \enskip right)~fake electron
	\label{fig:fakeLight_AIC_MLEPTONS}}
\end{center}
\end{figure}

For photons faking muons, we find better agreement if we apply a loss factor of 0.8 to the photon \pt when creating the fake trilepton sample, attributing an average of 20\,\% of the \pt to the lost lepton. If this factor is not applied, the peak location is not modeled accurately.
%Note, however, that the width of the background peak is not the same as in data (see finely binned $m_{3\ell}$ distribution in Fig.~\ref{fig:fakeLight_AIC_MLEPTONS_fine}).

%\begin{figure}
%\begin{center}
%	\includegraphics[width=.5\textwidth]{Background/bkg_fakeLight/AIC_MLIGHTLEPTONS_muFake_fine}%
%	\includegraphics[width=.5\textwidth]{Background/bkg_fakeLight/AIC_MLIGHTLEPTONS_elFake_fine}
%	\caption{$m_{3\ell}$ distribution in AIC-dominated control region, fine binning. \enskip left)~fake muon \enskip right)~fake electron
%	\label{fig:fakeLight_AIC_MLEPTONS_fine}}
%\end{center}
%\end{figure}

Outside the trilepton \Z window, it is necessary to increase the fake rate by 1.8 to achieve agreement. The plot shown in this Section have this factor applied.

We apply a 52\,\% systematic uncertainty on the total photon-based background estimate. This is because the photon fake rate depends on the environment flavor within this range.


\subsection{Fake leptons from jets}
\label{sec:bkg_fakeLight/jets}
For fake electrons and muons from jets, our proxies are isolated tracks in the 2$\ell$ data sample. We produce a track-based fake 3$\ell$ background seed sample by re-assigning isolated tracks to the lepton collections. All combinations are taken into account, \ie tracks are used to create both a fake-$e$ and a fake-$\mu$ event.\footnote{Multiple fakes in an event are not considered (neither of same proxy type (\eg two tracks) nor of different type). Hybrid fakes (one from a track, one from a photon) are currently not supported for technical reasons; same-type fakes however turned out to cause problems with the \ttbar MC subtraction. Given the smallness of the fake rates ($O(10^{-2})$), the contribution from multiple fakes is negligible anyways.}

We then look at events with 3 light leptons (no $\tau_\textrm{had}$) including an OSSF pair on \Z, no b-tags, and $\MET < 50\,\GeV$. This is the prominent \Z peak region with an additional lepton.

To make the \pt distributions match and thus achieve more accurate background modeling, we apply weights to the track-based background in bins of the the lowest \pt lepton (proxy) which is generally the fake.
%Numbers from 8 TeV: For the electron part, the weights are between 0.4\fixme{Explain that this is so high because of light jets, but leptons come via semileptonic decays from c/s etc.} and 2.3 (10..25\,\GeV), and 4.0 above 25\,\GeV; for the muons, we only need to scale the first bin (10..15\,\GeV) by 1.13. [Fig.~\ref{fig:fakeLight_Z_MINleptonPT}]

\begin{figure}
\begin{center}
	\includegraphics[width=.5\textwidth]{Background/bkg_fakeLight/Z_muFake_MINMUONPT}%
	\includegraphics[width=.5\textwidth]{Background/bkg_fakeLight/Z_elFake_MINELECTRONPT}
	\caption{\pt distributions of the lowest \pt lepton. \enskip left)~fake muon \enskip right)~fake electron
	\label{fig:fakeLight_Z_MINleptonPT}}
\end{center}
\end{figure}

We then measure the ratio of the number of 3$\ell$ and 2$\ell$+track events (fake rate), and investigate the dependence of this fake rate on the flavor of both the fake lepton and of the \Z decay products.
\begin{itemize}
	\item muons: $(1.53 \pm 0.22)\,\%$ ($ee$ environment), $(1.49 \pm 0.17)\,\%$ ($\mu\mu$ environment),
	\item electrons: $(1.38 \pm 0.22)\,\%$ ($ee$ environment), $(1.77 \pm 0.19)\,\%$ ($\mu\mu$ environment).
\end{itemize}
We find that the rates are consistent within statistical uncertainties, and therefore use one combined number only: $(1.56 \pm 0.10)\,\%$

The statistical uncertainty of the fake rate is taken as a systematic uncertainty. An additional systematic uncertainty of 10\,\% is assigned to account for inaccuracies from the \pt weighting.

To show that the MC subtraction described in the introduction to this chapter is valid, we need to verify that the $n_\textrm{tracks}$ distribution in the \ttbar MC sample matches the one in data, so that we can trust the fake rate method used in data is applicable for the \ttbar MC subtraction; see Section \ref{sec:bkg_tt}.

Fig.~\ref{fig:fakeLight_Z_MOSSF} shows the mass distribution of the ``best'' OS dilepton pair across its full range in the trilepton control region, both with and without an AIC veto (for off-\Z events whose 3$\ell$ invariant mass is on \Z). Fig.~\ref{fig:fakeLight_Z_MOSSF_byFlavor} distinguishes by flavors. In case of ambiguity, the ``best'' OS dilepton pair is the one whose invariant mass is closest to the \Z mass, with the additional condition that pairs above the \Z window are not considered if there is a pair below the \Z window (thus shifting events from high-\Z to low-\Z, for a more separative background categorization). For a comparison with another \Z-ness binning scheme, see Appendix~\ref{app:Zbinning}.

\begin{figure}
\begin{center}
	\includegraphics[width=.5\textwidth]{Background/bkg_fakeLight/Z_MOSSF}%
	\includegraphics[width=.5\textwidth]{Background/bkg_fakeLight/Z_noAIC_MOSSF}
	\caption{$m_{\ell\ell}$ distribution in the dilepton + fake region. \enskip left)~including events with $m_{\ell\ell\ell}$ on-\Z \enskip right)~events with $m_{\ell\ell\ell}$ on-\Z removed from the dilepton-off-\Z regions
	\label{fig:fakeLight_Z_MOSSF}}
\end{center}
\end{figure}

\begin{figure}
\begin{center}
	\includegraphics[width=.5\textwidth]{Background/bkg_fakeLight/Z_1el2mu_MOSSF}%
	\includegraphics[width=.5\textwidth]{Background/bkg_fakeLight/Z_3el_MOSSF}\\
	\includegraphics[width=.5\textwidth]{Background/bkg_fakeLight/Z_3mu_MOSSF}%
	\includegraphics[width=.5\textwidth]{Background/bkg_fakeLight/Z_2el1mu_MOSSF}\\
	\caption{$m_{\ell\ell}$ distribution in the dilepton + fake region. \enskip top)~$e$ fake \enskip bottom)~$\mu$ fake; \enskip left)~$\Z \to \mu\mu$ \enskip right)~$\Z \to e e$
	\label{fig:fakeLight_Z_MOSSF_byFlavor}}
\end{center}
\end{figure}

As an additional cross-check, we show the \HT distribution in the \Z region (Fig.~\ref{fig:fakeLight_Z_HT}).

\begin{figure}
\begin{center}
	\includegraphics[width=.7\textwidth]{Background/bkg_fakeLight/Z_HT}
	\caption{$\HT$ distribution in the dilepton fake region (no OSSF pair mass cut)
	\label{fig:fakeLight_Z_HT}}
\end{center}
\end{figure}

\section{\texorpdfstring{\ZZ}{ZZ} Background}
\label{sec:bkg_ZZ}

While the \ZZ background is responsible for only 3\,\% of the background in the ten channels with highest overall sensitivity, its contribution is around 40\,\% in the 4-lepton regions.

The control region is defined by 4 leptons, 2 OSSF pairs (at least one on-\Z), and $\MET < 50\GeV$. We use \ZZ MC with fully leptonic decays and normalize the total number of events in the control region, after subtracting other backgrounds. The normalization factor is $1.38 \pm 0.23\stat$. Fig.~\ref{fig:ZZ} shows the 4$\ell$ mass distribution in the control region.

\begin{figure}
\begin{center}
	\includegraphics[width=.7\textwidth]{Background/bkg_ZZ/ZZ_DYz2MET0to50HT0to200_MLIGHTLEPTONS}
	\caption{The $m_{4\ell}$ distribution in the \ZZ control region. The last bin is the overflow. Uncertainty bands include both statistical and systematic uncertainties, with the exception of the \ZZ normalization uncertainty.
	\label{fig:ZZ}}
\end{center}
\end{figure}


\chapter{Systematic Uncertainties}
\label{chap:Systematics}

Since most of the signal regions are limited by statistics, systematic uncertainties play a minor role. The only regions where we expect 10 or more events are the signal regions with 3 leptons including an OSSF pair on or above-\Z, and $L_\textrm{T} + \MET < 550\GeV$. In these regions, the \WZ and \ttbar background uncertainties become relevant. However, channels with higher $L_\textrm{T} + \MET$ are more sensitive to the signal. The full list of uncertainties is found in Table~\ref{tab:Systematics}, along with their impact on a representative set of three of the most sensitive channels.

\begin{table}
\centering
\small
\caption{Systematic uncertainties. The channels listed here have three leptons and $550\GeV < L_\textrm{T} + \MET < 750\GeV$.} \label{tab:Systematics}
\begin{tabular}{l c c c c}
\hline\hline
 & & \multicolumn{3}{c}{Impact on background/signal estimate in channel with} \\
Source of uncertainty & Magnitude & no OSSF pair & OSSF pair above-\Z & OSSF pair on-\Z \\
\hline
\WZ normalization                & 50\,\%       & 13\,\%  & 2.8\,\% & 41\,\%  \\
\ZZ normalization                & 16\,\%       & 0.1\,\% & 0.5\,\% & 0.4\,\% \\
Integrated luminosity            & 2.7\,\%      & 0.6\,\% & 0.2\,\% & 0.3\,\% \\
Lepton ID and isolation          &  3\,\%       & 3\,\%   & 3\,\%   & 3\,\%   \\
\MET resolution/smearing         & 50\,\%       & 4.1\,\% & 6.3\,\% & 0.6\,\% \\
Pile-up reweighting              & 5\,\%        & 1.5\,\% & 0.3\,\% & 1.3\,\% \\
\ttbar fake rate                 & 50\,\%       & 21\,\%  & 11\,\%  & 1.8\,\% \\
\Z + fake rate                   & 14\,\%       & 9.2\,\% & 1.1\,\% & 1.0\,\% \\
Rare MC cross section            & 50\,\%       & 11\,\%  & 2.7\,\% & 5.2\,\% \\
\\
Signal cross section             & 10\,\%       & 10\,\%  & 10\,\%  & 10\,\% \\
\\
\multicolumn{2}{l}{Total Background (for comparison)} & 0.3 events & 3.0 events & 3.5 events \\
\multicolumn{2}{l}{Signal ($m_\Sigma = 420\GeV$, for comparison)} & 0.8 events & 1.8 events & 0.8 events \\
\hline
\end{tabular}
\end{table}

The \ZZ and \ttbar uncertainties are based on the statistical uncertainties of the normalization regions; cross section uncertainties are thus not applied. For \WZ, we apply a 50\,\% uncertainty to account for the variation of the normalization factor depending on the \MET range chosen for normalization (see Sec.~\ref{sec:bkg_WZ}). For rare background processes, we apply a 50\,\% theory systematic uncertainty to cover both PDF as well as renormalization and factorization scale uncertainties. In the case of the signal, these uncertainties are covered by a 10\,\% systematic uncertainty \cite{CMS-PAS-EXO-14-001}.

For the \MET smearing procedure, a conservative uncertainty is determined by varying the amount of smearing by 50\,\%. Pile-up weights are evaluated by varying the minimzm-bias cross section by 5\,\% and propagating the impact on the pile-up weights through the analysis chain.

In general, systematic uncertainties are found by weighing events up or down or smearing them, then propagating those changes into the various bins of the analysis. The change in the expected backgrounds or signal yields in each bin corresponds to a systematic uncertainty, where we keep track of the relative sign of changes between different bins in order to take correlations and anti-correlations into account. Examples:
\begin{itemize}
	\item The luminosity uncertainty is correlated amongst all samples to which it is applied (i.\,e. MC samples that are not normalized to data). 
	\item As we apply the \MET smearing, events can migrate between $L_\textrm{T} + \MET$ bins. The uncertainty of the correction is thus anti-correlated between those bins. 
	\item An increase of the \WZ normalization by $1\sigma$ leads to a decrease of the measured \Z + jets fake rate, as we subtract \WZ background. Similarly, a $1\sigma$ increase of the \Z + jets fake rate leads to a decrease in the \WZ normalization, as the two control regions cannot be completely isolated from each other and have a (small) overlap. In all these cases, we take the relative signs of the changes into account to keep track of the correlations and anti-correlations.
\end{itemize}

Details on the fake rate uncertainties can be found in Sec.~\ref{sec:bkg_fakeLight}.

\chapter{Observation}
\label{sec:Results}

As described in the Section~\ref{sec:Optimization} above, the $L_\textrm{T} + \MET$ variable is a very efficient discriminator between the seesaw signal and the SM background. Therefore, the only requirement for the seesaw signal candidate events beyond the preliminary selection described in Section~\ref{sec:Selection} is that their $L_\textrm{T} + \MET$ value exceed 350\,\GeV. In Fig.~\ref{fig:Results} we present the $L_\textrm{T} + \MET$ distribution for four event categories as follows: 3 leptons with OSSF pair on-Z; 3 leptons with OSSF pair above-Z; 3 leptons with no OSSF pair; 4 leptons with at least one OSSF pair. Displays of the seesaw signal for heavy fermion mass $m_\Sigma = 420\,\GeV$ are also shown for each category. The signal generally stands out for higher values of $L_\textrm{T} + \MET$, as is to be expected for a massive parent particle. The SM background decomposition is also shown for each category.

\begin{figure}
\begin{center}
	\begin{subfigure}[b]{.5\textwidth}
		\includegraphics[width=\textwidth]{Results/plots/L3DYz1}
		\caption{3 leptons with OSSF pair on-\Z}
	\end{subfigure}%
	\begin{subfigure}[b]{.5\textwidth}
		\includegraphics[width=\textwidth]{Results/plots/L3DYh1}
		\caption{3 leptons with OSSF pair above-\Z}
	\end{subfigure}
	\begin{subfigure}[b]{.5\textwidth}
		\includegraphics[width=\textwidth]{Results/plots/L3DY0}
		\caption{3 leptons, no OSSF pair}
	\end{subfigure}%
	\begin{subfigure}[b]{.5\textwidth}
		\includegraphics[width=\textwidth]{Results/plots/L4DYgt0}
		\caption{4 leptons, at least 1 OSSF pair}
	\end{subfigure}%
	\caption{Results: $L_\textrm{T} + \MET$ distributions (last bin includes overflow in all plots).
	\label{fig:Results}}
\end{center}
\end{figure}

The observations are consistent with the SM expectations, with the possible exception of the 3-lepton category that includes an OSSF lepton pair with invariant mass consistent with that of the Z boson (Fig.~\ref{fig:Results}a). The dominant background for this category is the \WZ diboson production, as shown. The $p$-value for the hypothesis that the observation in the aggregated twenty $L_\textrm{T} + \MET$ bins for the four categories shown in Fig.~\ref{fig:Results} exceed SM expectation is 0.17. This overall consistency with the SM expectation conveys the message that the apparent excess of observed events in the top-left panel is either a statistical artifact or a discrepancy that can be addressed only with additional data.

\chapter{Conclusion}
\label{sec:Summary}

A search for type-III seesaw heavy fermion production has been performed in multilepton final states using \fullLumi of proton--proton collision data at $\sqrt{s} = 13\,\TeV$, collected using the CMS detector at the CERN LHC. No significant discrepancies between the background prediction and the data have been observed. Comparing the data with the predictions, we set upper limits at the 95\,\% confidence level on the production cross section of the heavy fermion pairs. Assuming degenerate heavy fermion masses $m_\Sigma$ in the flavor-democratic scenario, we exclude previously unexplored regions of the signal model with heavy fermion particle masses below $m_\Sigma < 435\,\GeV$ (expected: 430\,\GeV).


\clearpage

%%%%%%%%%%%%%%%%%% Appendix %%%%%%%%%%%%%%%%%%%%
\appendix
\chapter{Suitability of Tracks as Fake Proxies}
\label{app:MOSSFlepton,track}

We look at the invariant mass of opposite-sign muon + track pairs. If the tracks are uncorrelated with the muons, we expect a broad peak at the average invariant mass. If they are correlated, they will be related to their source, for example a \Z decay. In fact, we see both (Figure~\ref{fig:app:MOSSFlepton,track}).

\begin{sidewaysfigure}
\begin{center}
	\includegraphics[width=.5\textwidth]{Appendix/study_OSSFCLOSEMLL_electron,track_1fake}%
	\includegraphics[width=.5\textwidth]{Appendix/study_OSSFCLOSEMLL_electron,track_dileptons-1fake}\\
	\includegraphics[width=.5\textwidth]{Appendix/study_OSSFCLOSEMLL_muon,track_1fake}%
	\includegraphics[width=.5\textwidth]{Appendix/study_OSSFCLOSEMLL_muon,track_dileptons-1fake}
	\caption{Top: $m_{et}$ distribution; bottom: $m_{\mu t}$ distribution. Left: no other leptons present, right: additional lepton present. If a third lepton is present, its flavor is opposite of the first lepton, and its charge is the same as the track's (rejecting third leptons from \Z).\\
	\textbf{Note:} This is from the dilepton-triggered dataset, i.e. the tracks used here probably were good enough leptons to trigger.
	\label{fig:app:MOSSFlepton,track}}
\end{center}
\end{sidewaysfigure}

This suggests that tracks are not only from jets, but also from low quality leptons. So, tracks do model fake leptons from jets, and at the same time also model leptons that were vetoed by quality cuts. However, since both effects also occur in the signal regions, the overall shape of the track background is still expected to be accurate there.

Note that the numbers in Fig.~\ref{fig:app:MOSSFlepton,track} have been derived with the 8\,\TeV CMS dataset from Run I. The conclusions, however, are also valid for Run II at 13\,\TeV.

\chapter{Analysis Software}
\label{app:Software}

The analysis conducted in this thesis requires processing large amounts of data, both as additional datasets arrive from CERN, and also repeatedly as analysis needs develop. It is thus crucial that the software used to process the data performs efficiently, allowing the analysis to progress rapidly without too much computational delay. To achieve this, a three-tier ecosystem was developed in C++, based on standard CERN software. 

\section{RutgersAODReader}
The CMS experiment provides measurement data and simulated samples in the so-called MiniAOD format. This format can be decoded using the CMSSW software \cite{CMSSW} which is built on top of the ROOT data analysis framework \cite{Brun:1997pa} and also provided by CMS.

RutgersAODReader interfaces with CMSSW to extract event information from measurement data and simulated samples which is then used to create simple collections of objects that are associated with particles reconstructed by the detector. Those collections, also called ``products'', are stored along with a few basic analysis variables like \MET in so-called flat ntuple files using the standard ROOT file format.

\section{EventAnalyzer}
\label{app:Software/EventAnalyzer}
EventAnalyzer uses RutgersAODReader's flat ntuple output files as input. It then proceeds with higher-level computations, such as calculating isolation and other object-specific variables (see Sec.~\ref{sec:Selection/Object}), and creating collections of reconstructed particles with additional selection criteria such as collections of ``good muons'' and ``good electrons'' which fulfill the analysis requirements (isolation, promptness, etc.). The software also provides capabilities to make sure that objects do not overlap; for example, one can discard electrons that are too close to a reconstructed muon. Furthermore, particles can be reassigned to collections of another type, as is needed for the implementation of our fake rate method (see Sec.~\ref{sec:bkg_fakeLight/Method}). Fig.~\ref{fig:EventAnalyzer} displays an overview of the various object collections used.

\begin{figure}
\begin{center}
	\includegraphics[width=\textwidth]{Appendix/EventAnalyzer}
	\caption{Overview of the object collections used in EventAnalyzer.
	\label{fig:EventAnalyzer}}
\end{center}
\end{figure}

After calculating all the ``object variables'' which relate to a specific reconstructed particle, additional ``event variables'' are computed. Those are properties of the event and not of specific objects, e.\,g. the number of leptons passing all quality requirements, or whether an event contains an OSSF lepton pair whose invariant mass is consistent with a \Z boson decay (see Sec.~\ref{sec:Strategy/general}).

The categorization of data, background, and signal events into signal and control regions is usually based on such event variables. While EventAnalyzer provides several output mechanisms\hairspace{}---\hairspace{}from condensed histograms to full event dumps\hairspace{}---, the so-called AnalysisTree output format is used in most cases. This format is based on the standard ROOT file format and contains all computed event variables in a way that is easily accessible subsequently, either directly from the ROOT command prompt, or from within additional software that reads AnalysisTree files. Special storage mechanisms for boolean event variables and certain event weights are, however, incompatible with standard ROOT tools such as \texttt{hadd} (used for combining multiple files) and require custom variants of these tools (specifically, \texttt{haddR}).

All of EventAnalyzer's actions are configured through a configuration file. The EventAnalyzer engine itself is agnostic of the specific type of physics that is being investigated, but it understands concepts such as particle collections, momentum vectors, spatial distance, or invariant mass. The configuration file is used to specify which particles are to be related in which ways, and allows to declare variables for storage that are of interest on the analysis level such as the scalar sum of lepton transverse momenta, $L_\textrm{T}$. It also allows to specify the output format as well as skimming criteria in order to save on storage and processing time.

\section{AnalysisPresenter}
\subsection{Scope and Design Principles}
Once all data, background, and signal samples have been made available in AnalysisTree format, AnalysisPresenter is used to categorize events, apply event weights (such as pile-up, \pt, or $n_\textrm{jets}$ corrections), estimate the data-driven fake backgrounds by applying fake rates (including any parameterizations that may have been defined) to the seed samples (see Sec.~\ref{sec:bkg_fakeLight}), and scale MC backgrounds and signal to the data luminosity.

These scalings are performed on the fly at the time of reading events from the AnalysisTree input files. For each sample that is read in, a multidimensional histogram is created in memory which is filled using weights as prescribed by the scalings. Just like EventAnalyzer, the AnalysisPresenter engine is unaware of the physics that is being considered. Instead, the user needs to specify the variables of interest and the binning scheme through a configuration file. The axes of the multidimensional histogram are constructed based on these specifications and can later be used for cutting and binning. An axis can either be any event variable stored in the AnalysisTree, or a combination thereof. Simple combinations can be declared inline, while more complicated functions (such as a hash of the event number) require resorting to a user-defined C++ function from within the axis specification. Statistical uncertainties are stored with each bin that is populated.

AnalysisPresenter also allows modeling systematic uncertainties for backgrounds and signal. Correlations and anti-correlations can be modeled both between samples (e.\,g. the luminosity uncertainty for rare backgrounds estimated from MC, and for signal) as well as between bins within the same sample (e.\,g. bin migration due to \MET uncertainties). To keep track of these details for every sample, a copy of the multidimensional histogram is filled for each systematic uncertainty and sample, with a variation by one standard deviation applied. By subtracting the nominal histogram, the absolute impact of the systematic uncertainty can be determined; the sign distinguishes anti-correlations from correlations. Systematic uncertainties are specified either as a percentage number, as a formula that depends on arbitrary event variables or user-defined functions, or by declaring that an event variable be replaced by another one in order to evaluate the changes in how the multidimensional histogram is populated.

\subsection{Distributions}
Simple distributions like the $L_\textrm{T} + \MET$ distributions in Fig.~\ref{fig:Optimization2} can be made in a straightforward manner from the multidimenstional histograms: First, cuts are applied by setting an ``axis range'', e.g. $3 \leq n_\textrm{leptons} \leq 4$ to require three or four leptons. This can be thought of as slicing out a subset of the multidimensional histogram. Second, a projection is taken onto the axis of interest, e.\,g.  $L_\textrm{T} + \MET$. When projecting, the histogram is collapsed into a one-dimensional histogram along the projection axis; bins along all other dimensions are integrated. Statistical uncertainties are added in quadrature; systematic uncertainties are taken care of by repeating the slicing and projection steps with the correspondong copies of the multidimensional histograms.

The projection method returns an object that offers various ways to output the resulting distributions, for example as a plot like those in Fig.~\ref{fig:Optimization2}, or as an ASCII table. Within each projection object, an event list is stored along with the data, background, and signal histograms. This allows for easy investigation of interesting events.

\subsection{Bundling Mechanism and MC Subtraction}
As described in Sec.~\ref{sec:Samples/Signal}, the signal consists of 27 different processes which we generate separately. It is desirable that those samples are summed up and displayed together. Similarly, one might want the combine several rare backgrounds into one, to avoid visual clutter. For this purpose, AnalysisPresenter supports the bundling of samples. The bundling is performed after cutting and projecting has been done.

The bundling mechanism may not only be used for presentation purposes, but also for physical purposes. Our fake rate method, for example, requires subtracting an overlapping prediction from the \ttbar MC simulation. This can be handled by declaring the overlapping \ttbar background estimate as an additional background process in the AnalysisPresenter configuration file, but with a negative weight. This estimate can then be bundled with the fake background prediction itself; the negative weights for the \ttbar component will effect the desired subtraction. The result is what's labeled the ``TrackFakes'' component in Fig.~\ref{fig:fakeLight_Z_MOSSF}.

Samples of single physics processes share the same C++ interface as bundles of processes. This means that bundles may in turn be bundled up with other processes or bundles. For example, the ``Misidentified'' component in Fig.~\ref{fig:WZ} is a bundle of the ``TrackFakes'' and the ``PhotonFakes'' bundle. The summation is done for visual purposes only.

As we have seen, it is desirable to allow histogram bins to assume negative values for the purposes of MC subtraction through the bundling mechanism. Another use case of negative bin contents is when an MC generator such as MadGraph5\_aMC@NLO \cite{Alwall:2011uj} provides negative weights for some events. For bins in the tail of an event variable distribution, it may then happen that they are predominantly populated by events with negative weights. This is usually an artifact of the fact that the multidimensional histograms are extremely finely binned: Upon projection on any axis, the other axes are integrated over, so that the result is usually positive. It is thus important to keep the negative bins in the multidimensional histogram; discarding them would bias the histogram integral to higher average values when the binning scheme is replaced by a more granular one. Only after a projection is done, the user may want to enforce that negative bin contents be replaced by zero to make sure predictions are physical.

Thus, for each contribution (be it bundled or not), the user may specify whether bin contents should be bounded by 0 from below, depending on whether the contribution is intended to be used for subtraction purposes via the bundling mechansism or not.

\subsection{Channel Collections and Datacards}
A set of cuts (axis ranges) can be frozen into a ``channel''. Channels share the same C++ interface as projections; in particular, they can be output as a plot or a table.

In addition, multiple channels can be combined into a structure called ``channel collection''. From this structure, AnalysisPresenter can create a so-called datacard which contains information on observation, signal, background composition of all the channels involved, as well as statistical and systematic uncertainties including their correlations. The datacard format is compatible with the standard CMS statistics tools and can be used for statistical interpretation of the analysis, as is done in Sec.~\ref{sec:Interpretation}.

When creating a datacard, the bundling prescriptions are observed. This is not only important for purposes of MC subtraction, but also useful to limit the level of detail in the datacard in order to reduce the runtime of the statistical interpretation (e.\,g. to combine rare backgrounds). Furthermore, it is enforced that the channels within a channel collection do not have duplicate events, that is, they are exclusive.

\subsection{Runtime}
AnalysisPresenter is mainly a categorization tool, but does not actually calculate sophisticated physical quantities like invariant masses. (This is the task of EventAnalyzer, see Sec.~\ref{app:Software/EventAnalyzer}). It therefore can populate its main data structures\hairspace{}---\hairspace{}the multidimensional histograms\hairspace{}---\hairspace{}at a rate of about 20000 events per second. Nevertheless, the bulk of the processing time comes from reading events from the samples (between a few seconds and several minutes, depending on the use case).

Cuts, projections, bundlings, as well as channel collection and datacard creation are then performed within a fraction of second. Since it is possible to change the cuts at runtime, various types of output can be created for different selections without rereading the samples. Thus, if the user plans ahead and declares all axes that are intended to be used for cutting and binning, changes to the analysis selection can be studied very quickly in various regions. The runtime is roughly proportional to the number of samples and to the number of systematic uncertainties declared, and also depends on the binning granularity, as the bookkeeping overhead increases with the number of bins.



\clearpage

\addcontentsline{toc}{chapter}{Bibliography}
\begin{singlespace}
%\bibliographystyle{ieeetr}
\printbibliography
\end{singlespace}

%%%%%%%%%%%%%%%%%% End of thesis %%%%%%%%%%%%%%%%%%
%\pagestyle{empty} % Using this removes last page number of the bibliography for unknown reasons

\end{document}
